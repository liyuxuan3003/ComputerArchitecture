\section{单周期处理器}

单周期处理器的完整电路如\cref{fig:单周期RISC-V处理器}所示。当然,此刻这些电路看起来无比复杂,但本节会逐步通过各种角度,由浅入深的解释清楚这些电路是如何实现一个RISC-V处理器的功能的。

理解\cref{fig:单周期RISC-V处理器}的关键在于“1个ALU、2个ADD、3个MUX”,根据\cref{tab:RISC-V指令的功能分析}对指令的分析
\begin{itemize}
    \item ALU:指令执行需要进行不同的计算。LOAD/STORE需要通过加法计算\code{rs1+imm}确定对DMEM读写的地址。BRANCH指令需要通过减法计算\code{rs1-rs2}判定是否需要跳转。OP/OP-IMM则需要分别对\code{rs1,rs2}和\code{rs1,imm}进行加、减、与、或等。这些用途不同的计算都是由ALU完成,其接受两个输入\code{SrcA, SrcB},通过\code{ALUControl}选择计算类型,输出计算结果\code{ALUResult}的同时产生\code{Zero}用于指示结果是否为零。
    \item MUX(\code{SrcB}):指令的ALU计算只有两种情况,要么是\code{rs1,rs2},要么是\code{rs1,imm},因此,我们可以固定\code{SrcA}为\code{rs1},而在\code{SrcB}前连接MUX使其在\code{rs2}和\code{imm}间选择。
    \item MUX(\code{Result}):指令在执行完成后,多数情况需要将某个数值写回到\code{rd},这一数值有三种可能的来源,OP/OP-IMM来自ALU的计算结果\code{ALUResult},LOAD来自DMEM的读出数据\code{ReadData},JAL则来自\code{PCPlus4},因为使用\code{jal}进行跳转的同时,需要将当前指令顺序上的下一指令地址\code{pc+4}存入指定寄存器\code{rd},以便稍后能再回到这里。
    \item MUX(\code{PCNext}):指令大部分情况下是在顺序执行,应将\code{pc+4}即\code{PCPlus4}写入PC寄存器,作为下一周期的\code{pc},然而对于JAL和进行分支跳转的BRANCH,此时,应将目标地址\code{pc+imm}即\code{PCTarget}写入PC寄存器。故需在PC寄存器前设置一个MUX。
    \item ADD(\code{PCPlus4}):该加法器计算\code{pc+4},产生\code{PCPlus4},用于确定顺序执行的下一指令地址。请注意,即便在JAL下该加法器的输出同样有用,需要将\code{pc+4}写入\code{rd}。
    \item ADD(\code{PCTarget}):该加法器计算\code{pc+imm},产生\code{PCTarget},用于产生JAL和BRANCH的跳转指令地址。关于该加法器,一个常见的疑问是为什么其不能合并至ALU?这是因为BRANCH中ALU已被用于比较\code{rs1,rs2},故需专设一个加法器计算\code{pc+imm}。
\end{itemize}

\begin{Figure}[单周期RISC-V处理器]
    \includegraphics[width=0.47\textwidth]{Chapter07B_01.fig.pdf}
\end{Figure}
除此之外,在\cref{fig:单周期RISC-V处理器}中。Extend单元是用于将立即数扩展至$\qnum{32}{bit}$。参照\cref{fig:RISC-V指令类型},每一类指令编码立即数的位置和位数都有一定差异。Control Unit单元则是整个处理器的控制单元,通过对指令\code{op,funct3,funct7}的解码,产生MUX、ALU、REGFILE、DMEM等的控制信号。

\begin{Table}[RISC-V指令的功能分析]!!
    \begin{tblr}
    {
        colspec={lllX},
        cell{2-Z}{2-3}={preto=\ttfamily}
    }
        类型&指令&输入输出&作用\\
        I/LOAD&lw&rs1,imm,rd&由\texttt{rs1+imm} 的地址读取数据,数据写至\texttt{rd}\\
        S/STORE&sw&rs1,imm,rs2&向\texttt{rs1+imm} 的地址写入数据,数据读自\texttt{rs2}\\
        R/OP&add&rs1,rs2,rd&将\texttt{rs1,rs2} 作指定ALU运算,结果写至\texttt{rd}\\
        I/OP-IMM&addi&rs1,imm,rd&将\texttt{rs1,imm} 作指定ALU运算,结果写至\texttt{rd}\\
        B/BRANCH&beq&pc,imm,rs1,rs2&作\texttt{rs1-rs2},若结果为零,跳转至\texttt{pc+imm}\\
        J/JAL&jal&pc,imm,rd&在\texttt{rd} 中写入\texttt{pc+4},跳转至\texttt{pc+imm}\\
    \end{tblr}
\end{Table}

\subsection{单周期处理器的控制单元}
单周期处理器的控制单元如\cref{fig:单周期处理器的控制单元}所示,分为Main Decoder和ALU Decoder两级。
\begin{Figure}[单周期处理器的控制单元]
    \includegraphics[scale=0.8]{Chapter07B_08.fig.pdf}
\end{Figure}

\cref{tab:Main Decoder真值表}展示了Main Decoder的真值表,它以表征指令类型的\code{op[6:0]}为输入,产生了控制单元对外的大部分控制信号。不过,\code{ALUOp} 会传递给ALUDecoder结合\code{funct3,funct7}进一步译码,产生最终输出的\code{ALUControl}信号。这是因为,ALU的工作模式不仅仅取决于指令类型\code{op[6:0]},对于R/OP和I/OP-IMM类型,进行的计算和具体指令有关。\code{Branch,Jump}这两个信号在遇到B/BRANCH和J/JUMP类型时为高,它们共同产生了\code{PCSrc}信号。逻辑是,若\code{Jump}为高(无条件跳转),或\code{Branch}和\code{Zero}同时高(有条件跳转),\code{PCSrc}为高。
\begin{Table}[Main Decoder真值表]!!
    \begin{tblr}
    {
        colspec={lX*{8}{c}},
        cell{-}{2-Z}={preto=\ttfamily},
        column{3-Z}={leftsep=4pt,rightsep=4pt},
        column{Z}={leftsep=4pt},
    }
        类型&op&RegWrite&ImmSrc&ALUSrcB&MemWrite&ResultSrc&Branch&Jump&ALUOp\\
        I/LOAD & 0000011 & 1 & 00 & 1 & 0 & 1 & 0 & 0 & 00\\
        S/STORE & 0100011 & 0 & 01 & 1 & 1 & x & 0 & 0 & 00\\
        R/OP & 0110011 & 1 & xx & 0 & 0 & 0 & 0 & 0 & 10\\
        I/OP-IMM & 0010011 & 1 & 00 & 1 & 0 & 0 & 0 & 0 & 10\\
        B/BRANCH & 1100011 & 0 & 10 & 0 & 0 & xx & 1 & 0 & 01\\
        J/JAL & 1101111 & 1 & 11 & x & 0 & 10 & 0 & 1 & xx\\
    \end{tblr}
\end{Table}

\cref{tab:ImmSrc的控制编码}展示了Extend是如何在Main Decoder产生的\code{ImmSrc}下工作的。\code{ImmSrc}用两位编码了需要立即数的四种指令类型:I、S、B、J。Extend单元要做的就是根据\cref{fig:RISC-V指令类型}的各类型的立即数编码方式(编码已尽量保证较多的位数重合),正确的扩展出一个$\qnum{32}{bit}$的立即数。

\begin{Table}[ImmSrc的控制编码]!!
    \begin{tblr}
    {
        colspec={cX[2,c]X[12,l]r},
        cell{-}{2,3}={preto=\ttfamily},
    }
        类型&ImmSrc&ImmExt&说明\\
        I&00&\{20\{Instr[31]\},Instr[31:20]\}&$\qnum{12}{bit}$\\
        S&01&\{20\{Instr[31]\},Instr[31:25],Instr[11:7]\}&$\qnum{12}{bit}$\\
        B&10&\{20\{Instr[31]\},Instr[07],Instr[30:25],Instr[11:08],1'b0\}&$\qnum{13}{bit}$\\
        J&11&\{12\{Instr[31]\},Instr[19:12],Instr[20],Instr[30:21],1'b0\}&$\qnum{21}{bit}$\\
    \end{tblr}
\end{Table}

\cref{tab:ALU Decoder真值表}展示了ALU Decoder的真值表。注意到\code{ALUOp}信号的工作方式
\begin{itemize}
    \item 对于I/LOAD和S/STORE,\code{ALUOp=00},此时\code{ALUControl=000}做加法(地址偏移)。
    \item 对于B/BRANCH,\code{ALUOp=01},此时\code{ALUControl=001}做减法(比较)。
    \item 对于R/OP和I/OP-IMM,\code{ALUOp=10},此时\code{ALUControl}取决于具体指令的指定计算。
\end{itemize}

这里有一个较复杂的细节是,大部分R/OP和I/OP-IMM的指令之间都是通过\code{funct3}就能区分,然而\code{add,sub}的\code{funct3}是相同的,需要借助\code{funct7[5]=0,1}来区分,但更麻烦的事情在于,首先不存在\code{subi},其次\code{addi}没有\code{funct7}的字段,这一部分被其立即数编码占用了。注意到R/OP和I/OP-IMM的\code{op}分别为\code{0110011}和\code{0010011},可以通过\code{op[5]=0,1}区分,因此,当\code{op[5]=0}时,无论\code{funct7[5]}是什么(是立即数的某一位),都代表加法。

% {>{\ttfamily}X*{4}{@{\hspace{30pt}}>{\ttfamily}c}r}
\begin{Table}[ALU Decoder真值表]!!
    \begin{tblr}
    {
        colspec={lX[12,c]X[12,c]X[18,c]X[12,c]r},
        cell{1}{2-5}={preto=\ttfamily},
        cell{2-Z}{1-5}={preto=\ttfamily},
    }
        指令&ALUOp&funct3&\{op[5],funct7[5]\}&ALUControl&说明\\
        lw,sw & 00 & xxx & xx & 000 & 加法\\
        beq & 01 & xxx & xx & 001 & 减法\\
        add,addi & 10 & 000 & 00|01|10 & 000 & 加法\\
        sub & 10 & 000 & 11 & 001 & 减法\\
        slt,slti & 10 & 010 & xx & 101 & 当小于时赋值\\
        or,ori & 10 & 110 & xx & 011 & 按位或\\
        and,andi & 10 & 111 & xx & 010 & 按位与\\
    \end{tblr}
\end{Table}

\subsection{单周期处理器的数据通路}
首先,所有类型的指令共用的数据通路是:当前周期指令的地址\code{pc}从PC寄存器中取出,经过\code{PCPlus4}加法器计算顺序执行的下一条指令地址\code{pc+4}。同时,指令地址\code{pc}经过IMEM读出当前指令内容\code{instr},描述指令类型的\code{op,funct3,funct7}字段被送至控制单元译码,描述寄存器地址的\code{xrs1,xrs2,xrd}字段被送至寄存器堆的地址端,描述立即数的\code{ximm}字段被送至立即数扩展单元(这里前缀\code{x}用于区分寄存器地址和寄存器值/立即数字段和立即数)。

在LOAD和STORE指令类型中,ALU计算\code{rs1,imm}的加法,该计算结果的意义是DMEM地址\code{imm(rs1)},送至DMEM的\code{A}端。对于LOAD指令\code{MemWrite=0},DMEM从\code{RD}端读出数据\code{[imm(rs1)]}并经过MUX传递至REGFILE的\code{WD3}端处,从而将该数据写回至寄存器\code{rd}。对于STORE指令\code{MemWrite=1},DMEM自\code{WD}端接受写入\code{imm(rs1)}的数据\code{rs2}。

在OP和OP-IMM指令类型中,ALU对\code{rs1,rs2}和\code{rs1,imm}进行指令指定的计算,计算结果在“跳过DMEM”后直接通过MUX写回至\code{rd}。由此可见ALU输出的多重功能,对于写回值,既可能直接来自ALU的计算结果,也可能来自ALU的计算结果代表的地址处的数据。

在BRANCH和JAL指令类型中,\code{PCTarget}加法器计算跳转指令地址\code{pc+imm},\code{PCPlus4}加法器计算顺序指令地址\code{pc+4},两者都被送至产生的\code{PCNext}的MUX处,对于BRANCH指令,ALU会计算\code{rs1,rs2}的减法,ALU在正常输出外还会产生一个指示计算结果是否为零的的控制信号\code{Zero},其被接至Control Unit从而产生选择下一指令地址的\code{PCSrc}。而有所不同的是,对于JAL指令,其总是选择跳转地址,但\code{pc+4}同样不会浪费,其需要写回至\code{rd}。

\begin{Table}[单周期RISC-V处理器的数据通路]!!
    \begin{tblr}
    {
        colspec={X[c]X[c]},
        hline{2}={\linenone},
        hline{odd[2-Y]}={\linethin},
        vline{1,Z}={\linethick},
        vline{2}={\linethin},
        row{odd[1-Z]}={abovesep=6pt,belowsep=-4pt},
        row{even[1-Z]}={abovesep=0pt},
        rowhead=0,
    }
        \includegraphics[width=0.45\textwidth]{Chapter07B_02.fig.pdf}&
        \includegraphics[width=0.45\textwidth]{Chapter07B_03.fig.pdf}\\*
        I/LOAD&S/STORE\\
        \includegraphics[width=0.45\textwidth]{Chapter07B_04.fig.pdf}&
        \includegraphics[width=0.45\textwidth]{Chapter07B_05.fig.pdf}\\*
        R/OP&I/OP-IMM\\
        \includegraphics[width=0.45\textwidth]{Chapter07B_06.fig.pdf}&
        \includegraphics[width=0.45\textwidth]{Chapter07B_07.fig.pdf}\\*
        B/BRANCH&J/JAL\\
    \end{tblr}
\end{Table}

\subsection{单周期处理器的性能分析}
\cref{tab:参考电路延时}指出,在处理器中,存储器读取($t_{mem}=\qnum{200}{ps}$)和算数逻辑单元($t_{alu}=\qnum{120}{ps}$)是最为耗时的步骤,而LOAD类型的\code{lw}指令会同时涉及这两个操作(ALU计算地址,DMEM读取数据),作为最慢的指令,它会限制单周期处理器的最小周期。根据\cref{tab:单周期RISC-V处理器的数据通路}可知
\begin{Equation}!!
    T_{c,single}=t_{pcq}+t_{mem}+\max(t_{rf},t_{dec}+t_{ext}+t_{mux})+t_{alu}+t_{mem}+t_{mux}+t_{rf,setup}
\end{Equation}
传播时间$t_{pcq}$是上升沿到来之后多久信号能到达寄存器输出端。建立时间$t_{setup}$是上升沿到来前多久信号就要抵达寄存器输入端。因此,周期的计算,总是以某个寄存器的$t_{pcq}$起始,并且以某个寄存器的$t_{setup}$结束。其中,$t_{pc,pcq}$是新地址\code{pc}在上升沿后到达PC寄存器输出端的时间,$t_{mem}$ 是地址\code{pc}从IMEM中读取指令\code{instr}的时间。接下来,存在两条可能的关键路径:第一条$t_{rf}$是REGFILE中\code{rs1}读取,其连接到\code{SrcA}。第二条$t_{dec}+t_{ext}+t_{mux}$是\code{imm}的产生和传递,首先Control Unit解码出\code{ImmSrc},随后Extend在其控制下扩展出立即数\code{ImmExt},最后还需要经过MUX到达\code{SrcB}。随后,$t_{alu}$ 完成计算产生地址偏移\code{imm(rs1)}的时间,$t_{mem}$ 是DMEM读取出数据\code{[imm(rs1)]}的时间,$t_{mux}$ 是该数据通过MUX到达\code{Result}的时间,最后,$t_{rf,setup}$ 是REGFILE写回\code{rd}在下一周期上升沿到来前的建立时间。

经过化简,并根据\cref{tab:参考电路延时}注意到$t_{rf}=\qnum{100}{ps}$和$t_{dec}+t_{ext}+t_{mux}=\qnum{90}{ps}$
\begin{Equation}
    T_{c,single}=t_{pcq}+2t_{mem}+t_{rf}+t_{alu}+t_{mux}+t_{rf,setup}
\end{Equation}
将结果整理如下,并求出数值
\begin{BoxFormula}[单周期处理器的最小周期]
    单周期处理器的最小周期是
    \begin{Equation}
        T_{c,single}=t_{pcq}+2t_{mem}+t_{rf}+t_{alu}+t_{mux}+t_{rf,setup}=\qnum{750}{ps}
    \end{Equation}
\end{BoxFormula}
单周期处理器的$\te{CPI}$恒定为$1$,因为其任何指令都在一个周期内执行完成
\begin{BoxFormula}[单周期处理器的指令平均周期数]
    单周期处理器的CPI是
    \begin{Equation}
        \te{CPI}_{single}=1.00
    \end{Equation}
\end{BoxFormula}
单周期处理器在指令数$n=\qnum{1e11}{}$下的程序执行时间$T=n\cdot\te{CPI}\cdot T_{c}$因而为
\begin{BoxFormula}[单周期处理器的程序执行时间]
    单周期处理器的程序执行时间是
    \begin{Equation}
        T_{single}=\qnum{75}{s}
    \end{Equation}
\end{BoxFormula}