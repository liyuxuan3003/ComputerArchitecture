\section{单周期处理器}

单周期处理器的完整电路如\xref{fig:单周期RISC-V处理器}所示。当然,此刻这些电路看起来无比复杂,但本节会逐步通过各种角度,由浅入深的解释清楚这些电路是如何实现一个RISC-V处理器的功能的。

理解\xref{fig:单周期RISC-V处理器}的关键在于“1个ALU、2个ADD、3个MUX”,根据\xref{tab:RISC-V指令的功能分析}对指令的分析
\begin{itemize}
    \item ALU:指令执行需要进行不同的计算。LOAD/STORE需要通过加法计算\codex{rs1+imm} 确定对DMEM读写的地址。BRANCH指令需要通过减法计算\codex{rs1-rs2} 判定是否需要跳转。OP/OP-IMM则需要分别对\codex{rs1,rs2} 和\codex{rs1,imm} 进行加、减、与、或等。这些用途不同的计算都是由ALU完成,其接受两个输入\codex{SrcA, SrcB},通过\codex{ALUControl} 选择计算类型,输出计算结果\codex{ALUResult} 的同时产生\codex{Zero} 用于指示结果是否为零。
    \item MUX(\code{SrcB}):指令的ALU计算只有两种情况,要么是\codex{rs1,rs2},要么是\codex{rs1,imm},因此,我们可以固定\codex{SrcA} 为\codex{rs1},而在\codex{SrcB} 前连接MUX使其在\codex{rs2} 和\codex{imm} 间选择。
    \item MUX(\code{Result}):指令在执行完成后,多数情况需要将某个数值写回到\codex{rd},这一数值有三种可能的来源,OP/OP-IMM来自ALU的计算结果\codex{ALUResult},LOAD来自DMEM的读出数据\codex{ReadData},JAL则来自\codex{PCPlus4},因为使用\codex{jal} 进行跳转的同时,需要将当前指令顺序上的下一指令地址\codex{pc+4} 存入指定寄存器\codex{rd},以便稍后能再回到这里。
    \item MUX(\code{PCNext}):指令大部分情况下是在顺序执行,应将\codex{pc+4} 即\codex{PCPlus4} 写入PC寄存器,作为下一周期的\codex{pc},然而对于JAL和进行分支跳转的BRANCH,此时,应将目标地址\codex{pc+imm} 即\codex{PCTarget} 写入PC寄存器。故需在PC寄存器前设置一个MUX。
    \item ADD(\code{PCPlus4}):该加法器计算\codex{pc+4},产生\codex{PCPlus4},用于确定顺序执行的下一指令地址。请注意,即便在JAL下该加法器的输出同样有用,需要将\codex{pc+4} 写入\codex{rd}。
    \item ADD(\code{PCTarget}):该加法器计算\codex{pc+imm},产生\codex{PCTarget},用于产生JAL和BRANCH的跳转指令地址。关于该加法器,一个常见的疑问是为什么其不能合并至ALU?这是因为BRANCH中ALU已被用于比较\codex{rs1,rs2},故需专设一个加法器计算\codex{pc+imm}。
\end{itemize}

\begin{Figure}[单周期RISC-V处理器]
    \includegraphics[width=\linewidth]{build/Chapter07B_01.fig.pdf}
\end{Figure}
除此之外,在\xref{fig:单周期RISC-V处理器}中。Extend单元是用于将立即数扩展至$\SI{32}{bit}$。参照\xref{fig:RISC-V指令类型},每一类指令编码立即数的位置和位数都有一定差异。Control Unit单元则是整个处理器的控制单元,通过对指令\codex{op,funct3,funct7} 的解码,产生MUX、ALU、REGFILE、DMEM等的控制信号。

\begin{Tablex}[RISC-V指令的功能分析]{lllX}
    <类型&指令&输入输出&作用\\>
    I/LOAD&\texttt{lw}&\texttt{rs1,imm,rd}&由\texttt{rs1+imm} 的地址读取数据,数据写至\texttt{rd}\\
    S/STORE&\texttt{sw}&\texttt{rs1,imm,rs2}&向\texttt{rs1+imm} 的地址写入数据,数据读自\texttt{rs2}\\
    R/OP&\texttt{add}&\texttt{rs1,rs2,rd}&将\texttt{rs1,rs2} 作指定ALU运算,结果写至\texttt{rd}\\
    I/OP-IMM&\texttt{addi}&\texttt{rs1,imm,rd}&将\texttt{rs1,imm} 作指定ALU运算,结果写至\texttt{rd}\\
    B/BRANCH&\texttt{beq}&\texttt{pc,imm,rs1,rs2}&作\texttt{rs1-rs2},若结果为零,跳转至\texttt{pc+imm}\\
    J/JAL&\texttt{jal}&\texttt{pc,imm,rd}&在\texttt{rd} 中写入\texttt{pc+4},跳转至\texttt{pc+imm}\\
\end{Tablex}

\subsection{单周期处理器的控制单元}
单周期处理器的控制单元如\xref{fig:单周期处理器的控制单元}所示,分为Main Decoder和ALU Decoder两级。
\begin{Figure}[单周期处理器的控制单元]
    \includegraphics[scale=0.8]{build/Chapter07B_08.fig.pdf}
\end{Figure}

\xref{tab:Main Decoder真值表}展示了Main Decoder的真值表,它以表征指令类型的\codex{op[6:0]} 为输入,产生了控制单元对外的大部分控制信号。不过,\code{ALUOp} 会传递给ALUDecoder结合\codex{funct3,funct7} 进一步译码,产生最终输出的\codex{ALUControl} 信号。这是因为,ALU的工作模式不仅仅取决于指令类型\codex{op[6:0]},对于R/OP和I/OP-IMM类型,进行的计算和具体指令有关。\codex{Branch,Jump}这两个信号在遇到B/BRANCH和J/JUMP类型时为高,它们共同产生了\codex{PCSrc} 信号。逻辑是,若\codex{Jump} 为高(无条件跳转),或\codex{Branch} 和\codex{Zero} 同时高(有条件跳转),\codex{PCSrc} 为高。
\begin{Tablex}[Main Decoder真值表]{l>{\ttfamily}X*{8}{@{\hspace{6.5pt}}>{\ttfamily}c}}
    <类型&op&RegWrite&ImmSrc&ALUSrcB&MemWrite&ResultSrc&Branch&Jump&ALUOp\\>
    I/LOAD & 0000011 & 1 & 00 & 1 & 0 & 1 & 0 & 0 & 00\\
    S/STORE & 0100011 & 0 & 01 & 1 & 1 & x & 0 & 0 & 00\\
    R/OP & 0110011 & 1 & xx & 0 & 0 & 0 & 0 & 0 & 10\\
    I/OP-IMM & 0010011 & 1 & 00 & 1 & 0 & 0 & 0 & 0 & 10\\
    B/BRANCH & 1100011 & 0 & 10 & 0 & 0 & xx & 1 & 0 & 01\\
    J/JAL & 1101111 & 1 & 11 & x & 0 & 10 & 0 & 1 & xx\\
\end{Tablex}

\xref{tab:ImmSrc的控制编码}展示了Extend是如何在Main Decoder产生的\codex{ImmSrc} 下工作的。\codex{ImmSrc} 用两位编码了需要立即数的四种指令类型:I、S、B、J。Extend单元要做的就是根据\xref{fig:RISC-V指令类型}的各类型的立即数编码方式(编码已尽量保证较多的位数重合),正确的扩展出一个$\SI{32}{bit}$的立即数。

\begin{Tablex}[ImmSrc的控制编码]{l>{\ttfamily}c>{\ttfamily}Xr}
    <类型&ImmSrc&ImmExt&说明\\>
    I&00&\{20\{Instr[31]\},Instr[31:20]\}&\SI{12}{bit}有符号\\
    S&01&\{20\{Instr[31]\},Instr[31:25],Instr[11:7]\}&\SI{12}{bit}有符号\\
    B&10&\{20\{Instr[31]\},Instr[07],Instr[30:25],Instr[11:08],1'b0\}&\SI{13}{bit}有符号\\
    J&11&\{12\{Instr[31]\},Instr[19:12],Instr[20],Instr[30:21],1'b0\}&\SI{21}{bit}有符号\\
\end{Tablex}

\xref{tab:ALU Decoder真值表}展示了ALU Decoder的真值表。注意到\codex{ALUOp} 信号的工作方式
\begin{itemize}
    \item 对于I/LOAD和S/STORE,\codex{ALUOp=00},此时\codex{ALUControl=000} 做加法(地址偏移)。
    \item 对于B/BRANCH,\codex{ALUOp=01},此时\codex{ALUControl=001} 做减法(比较)。
    \item 对于R/OP和I/OP-IMM,\codex{ALUOp=10},此时\codex{ALUControl} 取决于具体指令的指定计算。
\end{itemize}

这里有一个较复杂的细节是,大部分R/OP和I/OP-IMM的指令之间都是通过\codex{funct3} 就能区分,然而\codex{add,sub} 的\codex{funct3} 是相同的,需要借助\codex{funct7[5]=0,1} 来区分,但更麻烦的事情在于,首先不存在\codex{subi},其次\codex{addi} 没有\codex{funct7} 的字段,这一部分被其立即数编码占用了。注意到R/OP和I/OP-IMM的\codex{op} 分别为\codex{0110011} 和\codex{0010011},可以通过\codex{op[5]=0,1} 区分,因此,当\codex{op[5]=0} 时,无论\codex{funct7[5]} 是什么(是立即数的某一位),都代表加法。

\begin{Tablex}[ALU Decoder真值表]{>{\ttfamily}X*{4}{@{\hspace{30pt}}>{\ttfamily}c}r}
    <\rmfamily 指令&ALUOp&funct3&\{op[5],funct7[5]\}&ALUControl&说明\\>
    lw,sw & 00 & xxx & xx & 000 & 加法\\
    beq & 01 & xxx & xx & 001 & 减法\\
    add,addi & 10 & 000 & 00|01|10 & 000 & 加法\\
    sub & 10 & 000 & 11 & 001 & 减法\\
    slt,slti & 10 & 010 & xx & 101 & 当小于时赋值\\
    or,ori & 10 & 110 & xx & 011 & 按位或\\
    and,andi & 10 & 111 & xx & 010 & 按位与\\
\end{Tablex}

\subsection{单周期处理器的数据通路}
首先,所有类型的指令共用的数据通路是:当前周期指令的地址\codex{pc} 从PC寄存器中取出,经过\codex{PCPlus4} 加法器计算顺序执行的下一条指令地址\codex{pc+4}。同时,指令地址\codex{pc} 经过IMEM读出当前指令内容\codex{instr},描述指令类型的\codex{op,funct3,funct7} 字段被送至控制单元译码,描述寄存器地址的\codex{xrs1,xrs2,xrd} 字段被送至寄存器堆的地址端,描述立即数的\codex{ximm} 字段被送至立即数扩展单元(这里前缀\codex{x} 用于区分寄存器地址和寄存器值/立即数字段和立即数)。

在LOAD和STORE指令类型中,ALU计算\codex{rs1,imm} 的加法,该计算结果的意义是DMEM地址\codex{imm(rs1)},送至DMEM的\codex{A} 端。对于LOAD指令\codex{MemWrite=0},DMEM从\codex{RD} 端读出数据\codex{[imm(rs1)]} 并经过MUX传递至REGFILE的\codex{WD3} 端处,从而将该数据写回至寄存器\codex{rd}。对于STORE指令\codex{MemWrite=1},DMEM自\codex{WD} 端接受写入\codex{imm(rs1)} 的数据\codex{rs2}。

在OP和OP-IMM指令类型中,ALU对\codex{rs1,rs2} 和\codex{rs1,imm} 进行指令指定的计算,计算结果在“跳过DMEM”后直接通过MUX写回至\codex{rd}。由此可见ALU输出的多重功能,对于写回值,既可能直接来自ALU的计算结果,也可能来自ALU的计算结果代表的地址处的数据。

在BRANCH和JAL指令类型中,\codex{PCTarget} 加法器计算跳转指令地址\codex{pc+imm},\codex{PCPlus4}加法器计算顺序指令地址\codex{pc+4},两者都被送至产生的\codex{PCNext} 的MUX处,对于BRANCH指令,ALU会计算\codex{rs1,rs2} 的减法,ALU在正常输出外还会产生一个指示计算结果是否为零的的控制信号\codex{Zero},其被接至Control Unit从而产生选择下一指令地址的\codex{PCSrc}。而有所不同的是,对于JAL指令,其总是选择跳转地址,但\codex{pc+4} 同样不会浪费,其需要写回至\codex{rd}。

\begin{TableLong}[单周期RISC-V处理器的数据通路]{|c|c|}*
    <\hlinemid>()
    \xcell<c>[1.5ex][0.0ex]{\includegraphics[width=0.47\linewidth]{build/Chapter07B_02.fig.pdf}}&
    \xcell<c>[1.5ex][0.0ex]{\includegraphics[width=0.47\linewidth]{build/Chapter07B_03.fig.pdf}}\\*
    I/LOAD&S/STORE\\ \hlinemid
    \xcell<c>[1.5ex][0.0ex]{\includegraphics[width=0.47\linewidth]{build/Chapter07B_04.fig.pdf}}&
    \xcell<c>[1.5ex][0.0ex]{\includegraphics[width=0.47\linewidth]{build/Chapter07B_05.fig.pdf}}\\*
    R/OP&I/OP-IMM\\ \hlinemid
    \xcell<c>[1.5ex][0.0ex]{\includegraphics[width=0.47\linewidth]{build/Chapter07B_06.fig.pdf}}&
    \xcell<c>[1.5ex][0.0ex]{\includegraphics[width=0.47\linewidth]{build/Chapter07B_07.fig.pdf}}\\*
    B/BRANCH&J/JAL\\ \hlinemid
\end{TableLong}

\subsection{单周期处理器的性能分析}
\xref{tab:参考电路延时}指出,在处理器中,存储器读取($t_{mem}=\SI{200}{ps}$)和算数逻辑单元($t_{alu}=\SI{120}{ps}$)是最为耗时的步骤,而LOAD类型的\codex{lw} 指令会同时涉及这两个操作(ALU计算地址,DMEM读取数据),作为最慢的指令,它会限制单周期处理器的最小周期。根据\xref{tab:单周期RISC-V处理器的数据通路}可知
\begin{Equation}
    \qquad
    T_{c,single}=t_{pc,pcq}+t_{mem}+\max(t_{rf},t_{dec}+t_{ext}+t_{mux})+t_{alu}+t_{mem}+t_{mux}+t_{rf,setup}
    \qquad
\end{Equation}
传播时间$t_{pcq}$是上升沿到来之后多久信号能到达寄存器输出端。建立时间$t_{setup}$是上升沿到来前多久信号就要抵达寄存器输入端。因此,周期的计算,总是以某个寄存器的$t_{pcq}$起始,并且以某个寄存器的$t_{setup}$结束。其中,$t_{pc,pcq}$是新地址\codex{pc} 在上升沿后到达PC寄存器输出端的时间,$t_{mem}$ 是地址\codex{pc} 从IMEM中读取指令\codex{instr} 的时间。接下来有两条可能的关键路径:第一条$t_{rf}$是REGFILE中\codex{rs1} 读取,其连接到\codex{SrcA}。第二条$t_{dec}+t_{ext}+t_{mux}$是\codex{imm} 的产生和传递,首先Control Unit解码出\codex{ImmSrc},随后Extend在其控制下扩展出立即数\codex{ImmExt},最后还需要经过MUX到达\codex{SrcB}。随后,$t_{alu}$ 完成计算产生地址偏移\codex{imm(rs1)} 的时间,$t_{mem}$ 是DMEM读取出数据\codex{[imm(rs1)]} 的时间,$t_{mux}$ 是该数据通过MUX到达\codex{Result} 的时间,最后,$t_{rf,setup}$ 是REGFILE写回\codex{rd} 在下一周期上升沿到来前的建立时间。

经过化简,并根据\xref{tab:参考电路延时}注意到$t_{rf}=\SI{100}{ps}$和$t_{dec}+t_{ext}+t_{mux}=\SI{90}{ps}$
\begin{Equation}
    T_{c,single}=t_{pc,pcq}+2t_{mem}+t_{rf}+t_{alu}+t_{mux}+t_{rf,setup}
\end{Equation}
将结果整理如下,并求出数值
\begin{BoxFormula}[单周期处理器的最小周期]
    单周期处理器的最小周期是
    \begin{Equation}
        \qquad\qquad\qquad
        T_{c,single}=t_{pc,pcq}+2t_{mem}+t_{rf}+t_{alu}+t_{mux}+t_{rf,setup}=\SI{750}{ps}
        \qquad\qquad\qquad
    \end{Equation}
\end{BoxFormula}
单周期处理器的$\te{CPI}$恒定为$1$,因为其任何指令都在一个周期内执行完成
\begin{BoxFormula}[单周期处理器的指令平均周期数]
    单周期处理器的CPI是
    \begin{Equation}
        \te{CPI}_{single}=1.0
    \end{Equation}
\end{BoxFormula}
单周期处理器在指令数$n=\SI{1e11}{}$下的程序执行时间$T=n\cdot\te{CPI}\cdot T_{c}$因而为
\begin{BoxFormula}[单周期处理器的程序执行时间]
    单周期处理器的程序执行时间是
    \begin{Equation}
        T_{single}=\SI{75}{s}
    \end{Equation}
\end{BoxFormula}