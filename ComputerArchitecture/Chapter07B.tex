\section{单周期处理器}

单周期处理器的完整电路如\xref{fig:单周期RISC-V处理器}所示。当然,此刻这些电路看起来无比复杂,但本节会逐步通过各种角度,由浅入深的解释清楚这些电路是如何实现一个RISC-V处理器的功能的。

理解\xref{fig:单周期RISC-V处理器}的关键在于“1个ALU、2个ADD、3个MUX”,根据\xref{tab:RISC-V指令的功能分析}对指令的分析
\begin{itemize}
    \item ALU:指令执行需要进行不同的计算。LOAD/STORE需要通过加法计算\codex{rs1+imm} 确定对DMEM读写的地址。BRANCH指令需要通过减法计算\codex{rs1-rs2} 判定是否需要跳转。OP/OP-IMM则需要分别对\codex{rs1,rs2} 和\codex{rs1,imm} 进行加、减、与、或等。这些用途不同的计算都是由ALU完成,其接受两个输入\codex{SrcA, SrcB},通过\codex{ALUControl} 选择计算类型,输出计算结果\codex{ALUResult} 的同时产生\codex{Zero} 用于指示结果是否为零。
    \item MUX(\code{SrcB}):指令的ALU计算只有两种情况,要么是\codex{rs1,rs2},要么是\codex{rs1,imm},因此,我们可以固定\codex{SrcA} 为\codex{rs1},而在\codex{SrcB} 前连接MUX使其在\codex{rs2} 和\codex{imm} 间选择。
    \item MUX(\code{Result}):指令在执行完成后,多数情况需要将某个数值写回到\codex{rd},这一数值有三种可能的来源,OP/OP-IMM来自ALU的计算结果\codex{ALUResult},LOAD来自DMEM的读出数据\codex{ReadData},JAL则来自\codex{PCPlus4},因为使用\codex{jal} 进行跳转的同时,需要将当前指令顺序上的下一指令地址\codex{pc+4} 存入指定寄存器\codex{rd},以便稍后能再回到这里。
    \item MUX(\code{PCNext}):指令大部分情况下是在顺序执行,应将\codex{pc+4} 即\codex{PCPlus4} 写入PC寄存器,作为下一周期的\codex{pc},然而对于JAL和进行分支跳转的BRANCH,此时,应将目标地址\codex{pc+imm} 即\codex{PCTarget} 写入PC寄存器。故需在PC寄存器前设置一个MUX。
    \item ADD(\code{PCPlus4}):该加法器计算\codex{pc+4},产生\codex{PCPlus4},用于确定顺序执行的下一指令地址。请注意,即便在JAL下该加法器的输出同样有用,需要将\codex{pc+4} 写入\codex{rd}。
    \item ADD(\code{PCTarget}):该加法器计算\codex{pc+imm},产生\codex{PCTarget},用于产生JAL和BRANCH的跳转指令地址。关于该加法器,一个常见的疑问是为什么其不能合并至ALU?这是因为BRANCH中ALU已被用于比较\codex{rs1,rs2},故需专设一个加法器计算\codex{pc+imm}。
\end{itemize}

\begin{Figure}[单周期RISC-V处理器]
    \includegraphics[width=\linewidth]{build/Chapter07B_01.fig.pdf}
\end{Figure}
除此之外,在\xref{fig:单周期RISC-V处理器}中。Extend单元是用于将立即数扩展至$\SI{32}{bit}$。参照\xref{fig:RISC-V指令类型},每一类指令编码立即数的位置和位数都有一定差异。Control Unit单元则是整个处理器的控制单元,通过对指令\codex{op,funct3,funct7} 的解码,产生MUX、ALU、REGFILE、DMEM等的控制信号。

\begin{Tablex}[RISC-V指令的功能分析]{lllX}
    <类型&指令&输入输出&作用\\>
    I/LOAD&\texttt{lw}&\texttt{rs1,imm,rd}&由\texttt{rs1+imm} 的地址读取数据,数据写至\texttt{rd}\\
    S/STORE&\texttt{sw}&\texttt{rs1,imm,rs2}&向\texttt{rs1+imm} 的地址写入数据,数据读自\texttt{rs2}\\
    R/OP&\texttt{add}&\texttt{rs1,rs2,rd}&将\texttt{rs1,rs2} 作指定ALU运算,结果写至\texttt{rd}\\
    I/OP-IMM&\texttt{addi}&\texttt{rs1,imm,rd}&将\texttt{rs1,imm} 作指定ALU运算,结果写至\texttt{rd}\\
    B/BRANCH&\texttt{beq}&\texttt{pc,imm,rs1,rs2}&作\texttt{rs1-rs2},若结果为零,跳转至\texttt{pc+imm}\\
    J/JAL&\texttt{jal}&\texttt{pc,imm,rd}&在\texttt{rd} 中写入\texttt{pc+4},跳转至\texttt{pc+imm}\\
\end{Tablex}

\subsection{单周期处理器的控制单元}
单周期处理器的控制单元如\xref{fig:单周期处理器的控制单元}所示,分为Main Decoder和ALU Decoder两级。
\begin{Figure}[单周期处理器的控制单元]
    \includegraphics[scale=0.8]{build/Chapter07B_08.fig.pdf}
\end{Figure}

\begin{Tablex}[Main Decoder真值表]{l>{\ttfamily}X*{8}{@{\hspace{6.5pt}}>{\ttfamily}c}}
    <类型&op&RegWrite&ImmSrc&ALUSrcB&MemWrite&ResultSrc&Branch&Jump&ALUOp\\>
    I/LOAD & 0000011 & 1 & 00 & 1 & 0 & 1 & 0 & 0 & 00\\
    S/STORE & 0100011 & 0 & 01 & 1 & 1 & x & 0 & 0 & 00\\
    R/OP & 0110011 & 1 & xx & 0 & 0 & 0 & 0 & 0 & 10\\
    I/OP-IMM & 0010011 & 1 & 00 & 1 & 0 & 0 & 0 & 0 & 10\\
    B/BRANCH & 1100011 & 0 & 10 & 0 & 0 & xx & 1 & 0 & 01\\
    J/JAL & 1101111 & 1 & 11 & x & 0 & 10 & 0 & 1 & xx\\
\end{Tablex}

\begin{Tablex}[ImmSrc的控制编码]{l>{\ttfamily}c>{\ttfamily}Xr}
    <类型&ImmSrc&ImmExt&说明\\>
    I&00&\{20\{Instr[31]\},Instr[31:20]\}&\SI{12}{bit}有符号\\
    S&01&\{20\{Instr[31]\},Instr[31:25],Instr[11:7]\}&\SI{12}{bit}有符号\\
    B&10&\{20\{Instr[31]\},Instr[07],Instr[30:25],Instr[11:08],1'b0\}&\SI{13}{bit}有符号\\
    J&11&\{12\{Instr[31]\},Instr[19:12],Instr[20],Instr[30:21],1'b0\}&\SI{21}{bit}有符号\\
\end{Tablex}

\begin{Tablex}[ALU Decoder真值表]{>{\ttfamily}X*{4}{@{\hspace{30pt}}>{\ttfamily}c}r}
    <\rmfamily 指令&ALUOp&funct3&\{op,funct7[5]\}&ALUControl&说明\\>
    lw,sw & 00 & xxx & xx & 000 & 加法\\
    beq & 01 & xxx & xx & 001 & 减法\\
    add,addi & 10 & 000 & 00|01|10 & 000 & 加法\\
    sub & 10 & 000 & 11 & 001 & 减法\\
    slt,slti & 10 & 010 & xxx & 101 & 当小于时赋值\\
    or,ori & 10 & 110 & xx & 011 & 按位或\\
    and,andi & 10 & 111 & xx & 010 & 按位与\\
\end{Tablex}

\newpage

\subsection{单周期处理器的数据通路}

\begin{TableLong}[单周期RISC-V处理器的原理分析]{|c|c|}*
    <\hlinemid>()
    \xcell<c>[1.5ex][0.0ex]{\includegraphics[width=0.47\linewidth]{build/Chapter07B_02.fig.pdf}}&
    \xcell<c>[1.5ex][0.0ex]{\includegraphics[width=0.47\linewidth]{build/Chapter07B_03.fig.pdf}}\\*
    I/LOAD&S/STORE\\ \hlinemid
    \xcell<c>[1.5ex][0.0ex]{\includegraphics[width=0.47\linewidth]{build/Chapter07B_04.fig.pdf}}&
    \xcell<c>[1.5ex][0.0ex]{\includegraphics[width=0.47\linewidth]{build/Chapter07B_05.fig.pdf}}\\*
    R/OP&I/OP-IMM\\ \hlinemid
    \xcell<c>[1.5ex][0.0ex]{\includegraphics[width=0.47\linewidth]{build/Chapter07B_06.fig.pdf}}&
    \xcell<c>[1.5ex][0.0ex]{\includegraphics[width=0.47\linewidth]{build/Chapter07B_07.fig.pdf}}\\*
    B/BRANCH&J/JAL\\ \hlinemid
\end{TableLong}

\subsection{单周期处理器的性能分析}