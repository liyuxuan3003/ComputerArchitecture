\section{流水线处理器}

流水线的思想和原理可以用“洗衣服”来形象解释。最初,我们使用的是一台洗烘一体机,衣服在完成清洗后会自动进行烘干,两者均需要半个小时。然而,我们注意到,清洗的半个小时烘干是闲置的,烘干的半个小时清洗是闲置的,这就是流水线能优化的空间所在。我们可以将其拆分为两台独立的洗衣机和烘干机,当有很多批衣服要清洗时,前一批衣服从洗衣机取出进入烘干机,后一批衣服就可以进入洗衣机了,同时有两批衣服在流水线上前进。这样平均下来,一批衣服完成清洗和烘干就只需要半个小时!当然,流水线的启动和结束需要额外时间。

流水线处理器的完整电路如\cref{fig:流水线RISC-V处理器}所示。它本质是将\cref{fig:单周期RISC-V处理器}的单周期处理器拆分为五个阶段,相互之间用寄存器间隔开,使得处理器能同时执行五条指令。流水线的五个阶段依次是
\begin{enumerate}
    \item 取指阶段(Fetch, IF)
    \item 译码阶段(Decode, ID)
    \item 执行阶段(Excute, EX)
    \item 存储阶段(Memory, MEM)
    \item 写回阶段(Writeback, WB)
\end{enumerate}
\begin{Figure}[流水线RISC-V处理器]
    \includegraphics[width=\linewidth]{Chapter07D_01.fig.pdf}
\end{Figure}

由于各个阶段内执行的是不同的指令,故在其信号后分别添加后缀\code{F,D,E,M,W}以示区分。

然而,流水线处理器的真正复杂之处并不在于流水线。关键是,指令和指令间是可能存在依赖关系的!若一条指令依赖于前序尚未执行完成的指令,这就会导致冒险(Hazard)的发生。冒险主要有两种类型:数据冒险(Data Hazard)是指,当前指令通过ALU计算或通过DMEM读取的数据,在后续几条指令中被立即使用,但此时数据尚未写回寄存器,从而导致后续指令读取了错误的寄存器数据。控制冒险(Control Hazard)是指,当前指令是JAL或BRANCH类型的跳转指令,尚未完成跳转决定和跳转地址的计算,从而导致后续指令被错误的顺序执行。冒险的正确识别和处理是流水线设计的关键!这也是\cref{fig:流水线RISC-V处理器}中Hazard Unit及围绕其的大量信号在做的事情。不过,在进一步详细讨论冒险之前,我们先有必要解释几处额外的修改。

第一处修改与\code{PCSrcE}有关。或许更明显的是,过去,产生\code{PCSrc}的与门和或门是位于控制单元内部的,而现在却暴露在外。这是因为,控制单元位于ID阶段,但是,产生\code{PCSrc}的输入之一\code{Zero}却要直到EX阶段才能由ALU给出,所以需要将逻辑门移动到控制单元外并置于EX阶段。我们可以看到,跳转决定\code{PCSrcE}和跳转地址\code{PCTargetE}在EX阶段产生,但却作用在IF阶段,这也是控制冒险的产生原因。在进行跳转前,已经有两条顺序执行指令错误进入了流水线。由此可见,控制冒险的处理的关键就在于从流水线中移除错误指令。

第二处修改与\code{RdW}有关。我们注意到,现在寄存器堆的三个输入中,前两者\code{A1,A2}是直接来自\code{InstrD}拆分出的\code{Rs1D,Rs2D},而\code{A3}却来自\code{RdW}。这是因为寄存器堆的写回地址\code{RdW}必须与写回数据\code{ResultW}同步,若使用\code{RdD}接入\code{A3},则会导致当前指令的写会地址被写入了三条指令之前的写回数据。类似的,写回地址\code{RdW}和写回数据\code{ResultW}的WB阶段和可能使用这些数据的ID阶段相差三条指令,这就是数据冒险的产生原因。若一条指令的写回数据需要被其后紧随的三条指令读取,那么读取到的是更新前的错误数据。不过,有一个简单的技巧可以让数据冒险的范围从“后三条”降低到“后两条”,这就是将寄存器堆的时钟由\code{CLK}更换为反相的 \texttt{\ctikztextnot{CLK}}。这会导致寄存器堆的写入发生在每个周期的中间而不是结束,换言之,寄存器堆将实现“前半周期写、后半周期读”。这样一来,由于相差三条指令时,后续指令和前续指令在同一周期分别于EX阶段和WB阶段读写寄存器堆,但由于先写后读,这就不再有冲突了。

% 有关于不同类型冒险的处理,我们有三种解决途径
% \begin{itemize}
%     \item 流水线前递(Forward),适用于数据冒险,其中数据来自ALU计算结果。
%     \item 流水线停滞(Stall),适用于数据冒险,其中数据来自DMEM读取结果。
%     \item 流水线冲刷(Flush),适用于控制冒险。
% \end{itemize}
% 实际上,这三种解决途径和冒险情形的对应关系只是大致上的,

\subsection{流水线前递}
流水线前递(Forward)的引入可以解决ALU计算结果导致的数据冒险,如\cref{fig:流水线前递}所示。这类图像可以很好的表现流水线不同阶段之间的依赖关系:横向代表同一条指令在不同周期依次经过的阶段,纵向代表同一周期下不同阶段的执行内容。这里的示意程序的关键在于,第一条指令由\code{t5,t6}相加得到\code{s0},第二第三第四条指令则分别令\code{s0}与\code{t1,t2,t3}依次做不同计算得到\code{s1,s2,s3},其中,对\code{s0}的立即使用导致在第二条和第三条指令上发生了数据冒险。

通过仔细观察,我们可以发现这里的数据冒险并不是因为结果没有被及时计算出来,只不过还没来得及写回寄存器。前递的思路是,通过额外路径,将刚刚计算出的“热乎”结果送到马上需要的位置。第二条和第三条指令在ID阶段从寄存器堆读取的\code{s0}都是错误的,然而,至第二条和第三条指令真正用到\code{s0}的EX阶段所在的周期T3和T4下时,正确的\code{s0}已经存在于第一条指令的MEM阶段和WB阶段。因此,只要能将\code{s0}跨阶段前递就能解决冒险。

在\cref{fig:流水线RISC-V处理器}中,注意到EX阶段原先\code{rs1,rs2}的位置添加了两个三通道的MUX,其中\code{00}连接的是从寄存器读出的当前指令的\code{rs1,rs2}的值\code{Rs1DataE,Rs2DataE},而\code{01}和\code{00}则分别是来自WB阶段和MEM阶段的前序指令的\code{rd}值,在当前指令中,它们可能被作为\code{rs1,rs2}使用。这里也关注到,Hazard Unit仅仅处理冒险相关的控制信号,而数据本身的传递并不经过它。这两个MUX的控制信号是\code{ForwardAE,ForwardBE},Hazard Unit通过输入的\code{Rs1E,Rs2E}和\code{RdM}及\code{RdW}比较,以判断两者是否要发生前递,并确定前递是来自MEM还是WB阶段。

\begin{Figure}[流水线前递]
    \includegraphics[width=\linewidth]{Chapter07D_02.fig.pdf}
\end{Figure}

综上,可以用伪代码的形式将\code{ForwardAE,ForwardBE}和相关输入的关系表示为
\begin{Code}[流水线前递的伪代码表示]{Verilog}
    \lstinputlisting{Chapter07D_01.cod.v}
\end{Code}

有两点需要解释。第一点是为何需要判断\code{RegWriteM}和\code{RegWriteW}?这是因为并非所有指令都有\code{rd}的字段,有些指令\code{rd}的位置是被立即数\code{imm}占用,我们必须确保判断\code{Rs1E,Rs2E}和\code{RdM,RdW}是否相等时\code{RdM,RdW}代表的是真正的\code{rd}而不是恰好碰上\code{imm}的部分字段,导致错误的前递。第二点是为何要判断\code{Rs1E,Rs2E}不为零。这是因为地址为\code{00000}的寄存器是零寄存器\code{zero},它常用于丢弃不需要的结果或作为恒定值\code{0}使用。然而,零寄存器不会造成数据冒险,不管写入什么数据,零寄存器读取到的始终是\code{0},故这种情况没有必要前递。

有一个问题是,为什么流水线前递只能解决ALU计算的数据冒险而不能解决DMEM读取的数据冒险?如\cref{fig:流水线前递无法用于DMEM读取导致的数据冒险}所示,第一条指令中\code{s0}的来源由\code{add}变为了\code{lw}。不同于\code{add}中MEM阶段只是原封不动的将EX阶段的计算结果向下传递,\code{lw} 中MEM阶段拿到的只是\code{s0}的地址\code{4(t5)},\code{lw}要直到WB阶段才有读取到的\code{s0}的值,这就导致紧随的指令无法前递,因为此时需要前递的结果确实没有产生。当然,我们会说,在T3周期不是明明已经有\code{s0}存在了吗,为什么不能前递?在这类图中,黑色字体表示从寄存器读出的数据,灰色字体表示经过该阶段处理之后的信号,如果将后者前递,意味着最小周期需要满足两个阶段的延时,具体的说,要满足DMEM读取和ALU计算的延时和,这是不合理的,故只有直接读出的数据可以前递。
 
\begin{Figure}[流水线前递无法用于DMEM读取导致的数据冒险]
    \includegraphics[width=\linewidth]{Chapter07D_03.fig.pdf}
\end{Figure}

\subsection{流水线停滞}
流水线停滞(Stall)的引入可以解决DMEM读取结果导致的数据冒险,如\cref{fig:流水线停滞}所示。简而言之,我们已经论证了DMEM读取的数据绝无可能在下一周期被使用,因此,当发生DMEM读取且下一条指令需要读取的数据时,我们必须要将流水线停滞一个周期。流水线停滞本质是对寄存器的干预,若在当前指令的EX阶段确认这是一条\code{lw}指令,且下一指令的\code{rs1,rs2}中有一个命中了当前指令的\code{rd},那就需要进行停滞。这包含两个部分:第一是停滞\code{PC}寄存器和\code{F/D}寄存器,使这两个寄存器在当前周期结束时不要更新,从而将\code{lw}后两条指令在原位停留一个周期。第二是冲刷\code{D/E}寄存器,使这个寄存器中存储的内容更换为一条\code{nop}之类的无效指令,从而避免\code{lw}后一条指令已经写入\code{D/E}寄存器的状态继续传播,造成额外的影响。
\begin{Figure}[流水线停滞]
    \includegraphics[width=\linewidth]{Chapter07D_04.fig.pdf}
\end{Figure}

这三者的控制信号分别是\code{StallF,StallD,FlushD}。在\cref{fig:流水线RISC-V处理器}中关注到\code{StallF,StallD}是添加反相之后连接到\code{PC}寄存器和\code{F/D}寄存器的\code{EN}端,这是因为“停滞”意味着“不使能”。

综上,可以用伪代码的形式将\code{StallF,StallD,FlushD}和相关输入的关系表示为
\begin{Code}[流水线停滞的伪代码表示]{Verilog}
    \lstinputlisting{Chapter07D_02.cod.v}
\end{Code}
其中,\code{ResultSrcE==01} 代表当前指令确实是\code{lw}指令,因为这意味着产生\code{Result}的MUX的输入源是\code{ReadData}。\code{(Rs1D==RdE)|(Rs2D==RdE)}代表\code{lw}指令的后一指令的两个输入中有一个用到了\code{lw}的\code{rd}。值的注意的是,这里并没有判断\code{rd}是否为零寄存器以及\code{rs2}并不是立即数的一部分,这有多方面的原因。首先,停滞和前递不同,错误的前递会导致不可接受的错误结果,错误的停滞只会浪费一个周期。其次,对于第一种情况,将数据读取到零寄存器并没有什么价值,不要编写这样的程序,对于第二种情况,立即数在\code{rs2}范围内的值代表的地址恰好碰上了上一条指令\code{rd}的地址的情况是罕见的,即便遇到,浪费一个周期也不要紧。

\subsection{流水线冲刷}
流水线冲刷(Flush)的引入可以解决控制冒险,如\cref{fig:流水线冲刷}所示。它的思路很简单,即便遇到涉及跳转的指令,也先顺序执行,这样若遇到\code{beq}但没有分支的情况,那就什么都不用做。而若遇到\code{beq}需要分支或是\code{jal}无条件跳转的情况,跳转决定和跳转地址会在EX阶段作出,那要做的就是将这期间错误顺序执行的两条指令冲刷掉即可,涉及\code{F/D}寄存器和\code{D/E}寄存器。
\begin{Figure}[流水线冲刷]
    \includegraphics[width=\linewidth]{Chapter07D_05.fig.pdf}
\end{Figure}
在EX阶段表征需要跳转的信号是\code{PCSrcE},因此就有
\begin{Code}[流水线冲刷的伪代码表示]{Verilog}
    \lstinputlisting{Chapter07D_03.cod.v}
\end{Code}

由于\code{FlushE}同时被停滞和冲刷两种机制使用,故其最终表达式应为两者的或。

\subsection{流水线处理器的性能分析}
流水线处理器每个阶段的耗时如下,需要说明的有两点:第一点是由于寄存器堆使用反相时钟,前半周期供WB阶段写,后半周期供ID阶段读,两者需要平分一个周期,因此这两个阶段所需的最小周期是其耗时的两倍,这也是为何ID阶段的耗时并不是以$t_{pcq}$开头。第二点是EX阶段的耗时中为何有$4t_{mux}$项?需要考虑最极端的情况,即需要发生分支跳转,且其第二项输入还是来自WB阶段的前递。这种情况下,首先\code{ReadDataW}需要通过WB阶段产生\code{ResultW}的MUX才能前递至EX阶段,随后经过\code{SrcBE}前的两个MUX进入ALU,而\code{Zero}信号还需要经过AND和OR两级逻辑门才能到达IF阶段\code{PCNextF}前的MUX的控制端,最后跳转地址\code{PCTargetE}本身还需要再通过这个MUX才能最终写入\code{PC}寄存器。
\begin{Gather}
    T_{c,pipelined,f}=t_{pcq}+t_{mem}+t_{setup}\\
    T_{c,pipelined,d}=2(t_{rf}+t_{setup})\\
    T_{c,pipelined,e}=t_{pcq}+4t_{mux}+t_{alu}+t_{and}+t_{or}+t_{setup}\\
    T_{c,pipelined,m}=t_{pcq}+t_{mem}+t_{setup}\\
    T_{c,pipelined,w}=2(t_{pcq}+t_{mux}+t_{rf,setup})
\end{Gather}

根据\cref{tab:参考电路延时},注意到EX阶段的延时最长,将结果整理如下
\begin{BoxFormula}[流水线处理器的最小周期]
    流水线处理器的最小周期是
    \begin{Equation}
        T_{c,pipelined}=t_{pcq}+4t_{mux}+t_{alu}+t_{and}+t_{or}+t_{setup}=\qnum{350}{ps}
    \end{Equation}
\end{BoxFormula}

流水线处理器的$\te{CPI}$理想状态下是$1$,然而,每个读取后立即使用会占用$2$个周期(停滞一个周期),每个跳转会占用$3$个周期(冲刷两条指令),结合\cref{tab:参考指令类型比例},可以算出实际的$\te{CPI}$为
\begin{BoxFormula}[流水线处理器的指令平均周期数]
    流水线处理器的$\te{CPI}$是
    \begin{Equation}
        \te{CPI}_{pipelined}=1.25
    \end{Equation}
\end{BoxFormula}
流水线处理器在指令数$n=\qnum{1e11}{}$下的程序执行时间$T=n\cdot\te{CPI}\cdot T_{c}$因而为
\begin{BoxFormula}[流水线处理器的程序执行时间]
    流水线处理器的程序执行时间是
    \begin{Equation}
        T_{pipelined}=\qnum{44}{s}
    \end{Equation}
\end{BoxFormula}
若比较\cref{fml:单周期处理器的程序执行时间}和\cref{fml:流水线处理器的程序执行时间},可以看出,相较于单周期处理器,流水线显著提高了效率!