\chapter{存储系统}

在\cref{chap:RISC-V处理器架构}中,我们假定处理器可以在一个周期内完成内存数据的访问,然而事实是,内存很慢!内存的读写速度要远远低于处理器的运行速度!通常而言,内存访问操作往往需要消耗处理器十个周期以上的时间(除非我们将处理器的时钟也降低到一个很慢的水平)。因此,如何利用缓存结构提高内存读写效率就成为一个相当重要的问题。本章中,首先会引入缓存设计遵循时间局域性和空间局域性的思想,随后会介绍“缓存-内存-硬盘”三级存储体系,了解不同存储器在结构、速度、容量、价格上的折中。最后会详细分析缓存的具体技术和实现方式。

\section{缓存原理与存储架构}

\subsection{缓存原理}

想像一个图书馆,而你在为毕业论文搜集资料。内存就像图书馆里一排又一排的书架,它们真的有很多书,但是穿过一排排书架找到一本想要的书需要花费很多时间。因此,我们会将几本正在看的书放在手边的桌子上。缓存就像图书馆里的桌子,只能放下有限的几本书,但是不用离开座位就可以翻阅。缓存设计最关键的问题就在于,我们应该将什么样的书放在桌上可以最大程度的减少我们离开座位去找书的次数?第一个想法称为时间局域性,如果我需要翻阅拉扎维的模拟CMOS集成电路,在未来的几个小时,我或许还要再次翻阅其中的章节,毕竟这本书非常晦涩难懂\footnote{如果你恰巧在学习模拟集成电路,艾伦的《CMOS模拟集成电路设计》比拉扎维的《模拟CMOS集成电路设计》有条理许多!},那不妨将这本书在桌上多放一会儿,直到桌上没有更多空间了。第二个想法称为空间局域性,如果我们准备从书架上拿了拉扎维的模拟CMOS集成电路,或许我们也会对同一个书架上奥本海姆的信号与系统很感兴趣,那不妨将这两本书一起抱回桌上。
\begin{itemize}
    \item 时间局域性(Temporal Locatity):若一个数据被访问,很有可能会再次访问该数据。
    \item 空间局域性(Spital Locatity):若一个数据被访问,很有可能会访问其临近数据。
\end{itemize}

缓存(Cache)的设计就是在利用数据访问行为的两个规律:时间局域性、空间局域性。

\subsection{缓存与层次化存储}

缓存的想法其实和存储器在速度和容量上的折中优化有关:我们无法制造出又快又大容量的存储器!但是我们可以制造“快但容量小”或“慢但容量大”的存储器,速度由快至慢有
\begin{itemize}
    \item 静态随机存取存储器(Static Random Access Memory, SRAM)
    \item 动态随机存取存储器(Dyanamic Random Access Memory, DRAM)
    \item 固态硬盘(Solid State Disk, SSD)
    \item 机械硬盘(Hard Drive Disk, HDD)
\end{itemize}

缓存的想法可以概括为:在越靠近CPU的位置使用越快但容量越小的存储器,使最有可能被访问的数据能在最短的时间被访问到。第一层是缓存(Cache),速度最快,它通常由SRAM构成,这是由六个晶体管构成的双稳态电路,在通电的状态下可以持久保持数据。Cache自身往往还有若干层,一般有L1 Cache和L2 Cache两层,容量通常为数百$\si{\kibi\byte}$和数$\si{\mebi\byte}$,可以在大致$1$个和$10$个周期内完成数据访问。Cache会和CPU做在同一块芯片上,从而最小化互连线的延时。第二层是内存(Memory),速度其次,它通常由DRAM构成,这是由一个晶体管和一个电容构成的电路,在通电的状态下需要定期刷新数据,否则会因为电容漏电丢失数据,容量通常为数十$\si{\gibi\byte}$,可以在约$\si{100}$个周期内完成数据访问。除此之外,硬盘作为非易失性存储,除了长期保存数据外,在有些情况下也可以作为虚拟内存(Virtual Memory)以内存交换的方式加入这个层次化存储体系,当然,作为必要的代价,它的速度就非常慢了。

\begin{Figure}[存储架构]
    \includegraphics[width=\linewidth]{MemoryStructure.fig.pdf}
\end{Figure}

\subsection{缓存的访问时间}

设缓存和内存的访问时间分别是$t_{cache}$和$t_{memory}$,若访问的数据存在于缓存中,称为缓存命中(Cache Hit),若访问的数据不存在于缓存中,称为缓存失效(Cache Miss),此时,需要进一步去内存中读取数据,假设缓存失效发生的概率记为$M$,则访问时间可以表示为
\begin{BoxFormula}[缓存的访问时间]
    若考虑缓存失效,访问缓存中数据所需的时间应为
    \begin{Equation}
        t=t_{cache}+Mt_{memory}
    \end{Equation}
\end{BoxFormula}

\section{缓存的实现方式}

缓存的结构可以用以下的描述来概括:整个缓存会被划分为$S$个组(Set),每个组包含$N$个块(Block),也称为$N$个通道或$N$路(Way),每个块包含$B$个字(Word)。因此,如果缓存的容量记为$C$,则显然有$C=S\cdot N\cdot B$成立。这里要澄清的是,通道数和(一个组中)块的数量是同一回事,通道中只有一个块!请始终记住当我们称一个缓存有$N$个通道或$N$路时就是在说一个组中有$N$个块。总结起来,可以用“$S$组$N$路且块大小为$B$”来描述一个缓存。
\begin{itemize}
    \item 组的数量$S$反映了缓存映射(从大至小)的取模运算的模数,这是缓存的基础。
    \item 路的数量$N$反映了缓存的时间局域性,将映射至同一组的数据缓存至不同块上。
    \item 块的大小$B$反映了缓存的空间局域性,将目标数据和邻近数据一起缓存。
\end{itemize}

\begin{Figure}[缓存的映射关系]
    \includegraphics[scale=0.7]{CacheMemoryMap.fig.pdf}
\end{Figure}

作为一个数据的参考,对于一个容量为$C=\qnum{64}{\Kibi}$的缓存,典型的通道数是$N=\qnum{2}{}$,典型的块大小是$B=\qnum{64}{}$。简洁起见,本节将考虑一个容量只有$C=8$的缓存,并依次分析三种情况
\begin{itemize}
    \item $S=8$,$N=1$,$B=1$,研究缓存映射的取模运算。
    \item $S=4$,$N=2$,$B=1$,研究缓存如何通过路的数量$N$利用时间局域性。
    \item $S=2$,$N=1$,$B=4$,研究缓存如何通过块的大小$B$利用空间局域性。
\end{itemize}

\subsection{缓存与取模运算}

缓存通过取模运算实现空间映射,如\cref{fig:缓存的映射关系}所示,对于容量$C=8$的缓存空间,我们可以将其分为$S=8$个组,每组存储$1$个数据,依次称为Set \code{000}、Set \code{001}、$\cdots$、Set \code{111}。而对于内存地址,除去因数据占用四个字节带来的最低$2$位固定\code{00}的Byte Offset后,根据次低$3$位映射至Set编号相同的缓存组中,如\code{00...000 00100}、\code{00...001 00100}、\code{11...111 00100}都可以映射到\code{001}的缓存处。这样一来,如果重复读取同一个数据,读取速度就可以大大加快!

缓存是如何通过地址判断正在读取的数据是否存在于缓存中呢?如\cref{fig:缓存与取模运算}所示,内存地址从低到高被划分为Byte Offset、Set、Tag三个字段,缓存除了存储数据,还要存储数据对应地址的Tag字段,作为区分存储的是映射到该位置的哪一数据的标签。读取时,首先根据读取地址的Set字段索引出缓存中的对应行,随后比较读取地址和缓存中的Tag字段,若一致,说明这是一次Cache Hit,反之,说明这是一次Cache Miss,需要进一步去内存中读取数据。另外,缓存还需要维护一个额外的V字段,用来标识该Set为空,这在刚初始化后是有用的。

\begin{Figure}[缓存与取模运算]
    \includegraphics[scale=0.7]{CacheSet.fig.pdf}
\end{Figure}

\subsection{缓存与时间局域性}

缓存如何实现时间局域性?试想,若交替读取两个Set一致的数据,那么缓存将无法发挥任何作用!如\cref{fig:缓存与时间局域性}所示,可以只使用$S=4$组,每组中设置$W=2$路,每路中存储$B=1$个数据。同一组中存在多路的好处是,同一个Set可以容纳多个不同Tag的数据,从而使最近使用的数据能停留更长时间!我们通过一个简单的U字段来判断哪一路应该优先被覆盖,例如这里刚刚位于Way \code{1}上的数据,那么就要将U设置为\code{0}以表示接下来应优先覆盖Way \code{0}上的数据。不过,值得注意的是,如果存在更多路,一般仍然会使用$1$位的U字段!我们会将所有的路分为两个部分,用U标识最近没有使用的那一部分,当需要缓存新数据时,随机在其中选择一路进行覆盖。这种方法相对简单,且在绝大多数情况下足矣提供足够的时间局域性。

\begin{Figure}[缓存与时间局域性]
    \includegraphics[scale=0.7]{CacheWay.fig.pdf}
\end{Figure}

\subsection{缓存与空间局域性}

缓存如何实现空间局域性?试想,若连续读取数据,比较高效的方法是将其邻近的数据一并读取回来。如\cref{fig:缓存与空间局域性}所示,可以只使用$S=2$组,简洁起见,每组中仅设置$W=1$路,但现在每路中可以存储$B=4$个数据。在Set之外新增了一个Block字段,当读取某个数据时,会同时将邻近的Block为\code{00}、\code{01}、\code{10}、\code{11}的数据读至缓存,这就增加了空间局域下的缓存命中率。

\begin{Figure}[缓存与空间局域性]
    \includegraphics[scale=0.7]{CacheBlock.fig.pdf}
\end{Figure}

然而,一个值得思考的问题是,如果访问内存的时间是一定的,那如果每次都要读取$4$个数据,是否需要消耗$4$倍的时间,使利用空间局域性的尝试变得没有意义?答案是否定的,因为存储器往往支持这样的读取模式,提供一个地址,要求其连续传输从这个地址起的一系列数据。第一个数据到达需要若干周期,但接下来每个周期都能传输一个数据,故不会增加太多负担。

空间局域性和时间局域性是可以并存的!每组中可以有若干路,每路中可以存储若干数据。

\subsection{缓存的写策略}

上述主要讨论的是缓存的读取,缓存的写入基本是类似的,但会有一点不同:当数据写入缓存后,以何种方式同步到内存中?这衍生出两种策略:写直通(Write Through)和写回(Write Back)。写直通策略意味着数据在写入缓存后也立即写入内存,这会导致写数据总是需要消耗较长时间,但是比较简单。写回策略意味着数据在写入缓存后暂时不写入内存,而是在缓存中将该数据标记为脏(Dirty Bit),当包含脏数据的缓存要被覆盖时,再将数据写回到内存中。