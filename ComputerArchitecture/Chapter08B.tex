\section{缓存的实现方式}

缓存的结构可以用以下的描述来概括:整个缓存会被划分为$S$个组(Set),每个组包含$N$个块(Block),也称为$N$个通道或$N$路(Way),每个块包含$B$个字(Word)。因此,如果缓存的容量记为$C$,则显然有$C=S\cdot N\cdot B$成立。这里要澄清的是,通道数和(一个组中)块的数量是同一回事,通道中只有一个块!请始终记住当我们称一个缓存有$N$个通道或$N$路时就是在说一个组中有$N$个块。总结起来,可以用“$S$组$N$路且块大小为$B$”来描述一个缓存。
\begin{itemize}
    \item 组的数量$S$反映了缓存映射(从大至小)的取模运算的模数,这是缓存的基础。
    \item 路的数量$N$反映了缓存的时间局域性,将映射至同一组的数据缓存至不同块上。
    \item 块的大小$B$反映了缓存的空间局域性,将目标数据和邻近数据一起缓存。
\end{itemize}

作为一个数据的参考,对于一个容量为$C=\qnum{64}{\Kibi}$的缓存,典型的通道数是$N=\qnum{2}{}$,典型的块大小是$B=\qnum{64}{}$。简洁起见,本节将考虑一个容量只有$C=8$的缓存,并依次分析三种情况
\begin{itemize}
    \item $S=8$,$N=1$,$B=1$,研究缓存映射的取模运算。
    \item $S=4$,$N=2$,$B=1$,研究缓存如何利用时间局域性。
    \item $S=2$,$N=1$,$B=4$,研究缓存如何利用空间局域性。
\end{itemize}

\subsection{缓存与取模运算}

\subsection{缓存与时间局域性}

\subsection{缓存与空间局域性}