\documentclass{xStandalone}

\begin{document}
\begin{tikzpicture}[font=\ttfamily]

\path (0,0)  coordinate (xL)
    ++(26,0) coordinate (xR);

\path (0,-2) coordinate (yC);

\path (0,-8.25) coordinate (yB);
\path (0,10.00) coordinate (yA);

\draw[ultra thin] (xL|-yB) rectangle (xR|-yA);

% MEM
\path (xL|-yC) ++(4.3,0) node[mem,anchor=west] (MEM) {Memory};

% REGFILE
\path (MEM.east) ++(4.25,0) node[regfile,anchor=west] (REGFILE) {Register\\ File};

% Marker on MEM/REGFILE
\foreach \block/\n/\side/\text in
{%
    MEM/1/west/A,
    MEM/1/east/RD,
    MEM/4/west/WD,
    REGFILE/1/west/A1,
    REGFILE/2/west/A2,
    REGFILE/3/west/A3,
    REGFILE/4/west/WD3,
    REGFILE/1/east/RD1,
    REGFILE/2/east/RD2%
}%
{
    \path ($(\block.north \side)!\fpeval{0.2+0.2*(\n-1)}!(\block.south \side)$) coordinate (\block -\text) node[anchor=\side] {\text};
}

% ---- Around MEM ---- %

% MUX Adr
\path (MEM-A) ++ (-0.75,0) node[mux21,anchor=east] (MUXAdr) {};

% REG PC
\path (xL|-MUXAdr.blpin 1) ++(1.5,0) node[register,anchor=west] (REGPC) {};

% REG Instr
\path (MEM.north east) ++(1.25,0.1) node[register,minimum height=3.0cm,anchor=west] (REGInstr) {};
\path (MEM-RD) coordinate (yREGInstr2);
\path ($(yREGInstr2)!2.0!(REGInstr)$) coordinate (yREGInstr1);

% REG Data
\path (REGInstr.south|-MEM.south) ++(0,-2.0) node[register,anchor=north] (REGData) {};

% ---- Around REGFILE ---- %

% IMM Extend
\path (REGFILE.south) ++ (0,-1.5) node[extend,anchor=north] (IMMExtend) {Extend};

% REG RD
\path ($(REGFILE-RD1)!0.5!(REGFILE-RD2)$) ++(1,0) node[register,minimum height=2.0cm,anchor=west] (REGRD) {};
\path (REGFILE-RD1) coordinate (yREGRD1);
\path (REGFILE-RD2) coordinate (yREGRD2);

% MUX SrcA
\path (REGRD.west|-yREGRD1) ++ (2.0,0) node[mux31,anchor=blpin 3] (MUXSrcA) {};

% MUX SrcB
\path (REGRD.west|-yREGRD2) ++ (2.8,0) node[mux31,anchor=blpin 1] (MUXSrcB) {};

% ALU
\path (MUXSrcB.brpin 1|-MUXSrcA.brpin 1) ++ (1.2,0) node[add,anchor=blpin 1] (ALU) {\rotatebox{90}{ALU}};

% REGALUOut
\path (ALU.brpin 2) ++ (2.0,0) node[register,anchor=west] (REGALUOut) {};

% MUX Result
\path (xR|-REGALUOut.east) ++(-1.5,0) node[mux31,anchor=blpin 1] (MUXResult) {};

% Control Unit
\path (REGFILE.north west) ++ (0,3.5) node[control,anchor=south west] (ControlUnit) {Control\\ Unit};

% ---- Control Unit Signal ----%
\foreach \n/\side/\text/\label in
{
    1/west/PCWrite/PCWrite,
    2/west/AdrSrc/AdrSrc,
    3/west/MemWrite/MemWrite,
    4/west/IRWrite/IRWrite,
    1/east/ResultSrc[1:0]/ResultSrc,
    2/east/ALUControl[2:0]/ALUControl,
    3/east/ALUSrcB[1:0]/ALUSrcB,
    4/east/ALUSrcA[1:0]/ALUSrcA,
    5/east/ImmSrc[1:0]/ImmSrc,
    6/east/RegWrite/RegWrite%
}
{
    \ifthenelse{\equal{\side}{west}}{\def\direction{above left}}{\def\direction{above right}}
    \path ($(ControlUnit.north \side)!\fpeval{(0.5/9)+(1/9)*(\n-1)}!(ControlUnit.south \side)$) coordinate (CU-\label) node[\direction,inner ysep=0.4ex] {\text\vphantom{hg}};
}

\foreach \n/\side/\text/\label in
{
    6/west/op/op,
    7/west/funct3/funct3,
    8/west/funct7/funct7,
    9/west/Zero/Zero%
}
{
    \path ($(ControlUnit.north \side)!\fpeval{(0.5/9)+(1/9)*(\n-1)}!(ControlUnit.south \side)$) coordinate (CU-\label) node[right] {\text};
}

% ---- Clock Mark ---- %
\foreach \block/\x in
{
    REGPC/0.5,
    REGInstr/0.5,
    REGRD/0.5,
    REGData/0.5,
    REGALUOut/0.5,
    REGFILE/0.33,
    MEM/0.33%
}
{
    \path ($(\block.north west)!\x!(\block.north east)$) node[clk] (\block -CLK) {};
    \draw[line control] (\block -CLK) -- ++(0,0.5) node[ocirc] {} node[above] {CLK};
}

% ---- Clock Mark ---- %
\foreach \block/\x in
{
    REGPC/0.5,
    REGInstr/0.5
}
{
    \path ($(\block.south west)!\x!(\block.south east)$) node[above] {EN};
}

% ---- WE Mark ---- %
\foreach \block/\x in
{
    REGFILE/0.67,
    MEM/0.67%
}
{
    \path ($(\block.north west)!\x!(\block.north east)$) coordinate (\block -WE) node[below] {WE};
}

% ---- Mux21 Mark ---- %
\foreach \mux in {MUXAdr}
{
    \path (\mux.blpin 1) node[muxopt] {0};
    \path (\mux.blpin 2) node[muxopt] {1};
}

% ---- Mux31 Mark ---- %
\foreach \mux in {MUXResult,MUXSrcA,MUXSrcB}
{
    \path (\mux.blpin 1) node[muxopt] {00};
    \path (\mux.blpin 2) node[muxopt] {01};
    \path (\mux.blpin 3) node[muxopt] {10};
}


% ---- X ---- %
\path (REGPC.west) ++(-0.75,0) coordinate (xREGPCin);

\path (MUXAdr.west) ++(-0.35,0) coordinate (xPC) coordinate (xResultAdr);

\path (MEM.west) ++(-0.5,0) coordinate (xWriteDataWB);

\path ($(MEM.east)!0.5!(REGInstr.west)$) coordinate (xReadData);

\path (REGPC.west) ++(-0.3,0) coordinate (xREGPCEN);

\path (REGInstr.west) ++(-0.3,0) coordinate (xREGInstrEN);

\path ($(MEM.east)!0.78!(REGFILE.west)$) coordinate (xInstr) ++(-0.35,0) coordinate (xResultWD3);

\path (REGFILE.east) ++(0.5,0) coordinate (xImmSrc);

\path (REGRD.east) ++(0.25,0) coordinate (xWriteData) ++(0.5,0) coordinate (xIMMExt);

\path (MUXSrcA.blpin 1) ++ (-0.50,0) coordinate (xMUXSrcAin1);

\path (MUXSrcA.blpin 2) ++ (-1.00,0) coordinate (xMUXSrcAin2);

\path (MUXSrcB.blpin 3) ++ (-0.50,0) coordinate (xMUXSrcBin3);

\path (MUXSrcB.brpin 1) ++(0.25,0) coordinate (xSrc);

\path (ALU.east) ++ (1.5,0) coordinate (xZero);

\path (ControlUnit.west) ++(-0.25,0) coordinate (xZeroCU);

\path (REGALUOut.west) ++(-0.50,0) coordinate (xALUResult);

\path (MUXResult.west) ++(-1.00,0) coordinate (xData);

\path (MUXResult.east) ++(0.25,0) coordinate (xResult);

% ---- Y ---- %
\path (REGPC.south) ++(0,-0.25) coordinate (yREGPCEN);

\path (REGInstr.south) ++(0,-0.25) coordinate (yREGInstrEN);

\path (REGData.south) ++(0,-0.25) coordinate (yResultWB);

\path (IMMExtend.north east) ++(0,0.5) coordinate (yWriteDataWB) ++(0,0.25) coordinate (yImmSrc);

\path (ControlUnit.south) ++(0,-0.50) coordinate (yZeroCU) ++(0,-0.35) coordinate (yPC);

% ---- Line Data ---- %

% REG ->[PC] MUXAdr
\draw[emph data] (REGPC.east) -- node[above,pos=0.3,signal] {PC} node[below,value] {pc} (MUXAdr.blpin 1);

% MUXAdr ->[Adr] MEM.A
\draw[emph data] (MUXAdr.brpin 1) -- node[above,signal] {Adr} node[below,value] {pc} (MEM-A); 

% PC -> REGInstr
\draw[emph data] (xPC|-REGPC.east) node[circ] {} |- (REGInstr.west|-yREGInstr1) node[above left,xshift=-0.5cm,value] {pc};

% PC -> MUXSrcA.in1
\draw[emph data] (xPC|-yREGInstr1) node[circ] {} |- (xMUXSrcAin1|-yPC) node[below left,value] {pc} |- (MUXSrcA.blpin 1);

% MEM.RD -> REGInstr
\draw[emph data] (MEM-RD) -- node[above,value,pos=0.4] {instr} (REGInstr.west|-MEM-RD);

% MEM.RD ->[ReadData] REGData
\draw[line data] (xReadData|-MEM-RD) node[circ] {} -- node[right,pos=0.4,signal] {\rotatebox{-90}{ReadData}} (xReadData|-REGData.west) -- (REGData.west);

% MEM -[Instr] |
\draw[line data] (REGInstr.east|-yREGInstr2) node[above right,signal] {Instr} --  (xInstr|-yREGInstr2);
\draw[line data] (xInstr|-CU-op) -- (xInstr|-IMMExtend.west);

% MEM Split
\foreach \y/\bmax/\bmin in {CU-op/06/00,CU-funct3/14/12,CU-funct7/31/25,REGFILE-A1/19/15,REGFILE-A2/24/20,REGFILE-A3/11/07,IMMExtend.west/31/07}
{
    \draw[line data] (xInstr|-\y) node[above right,signal] {\scalebox{0.5}{[\bmax :\bmin]}} -- (\y);
}

\foreach \y in {CU-funct3,CU-funct7,REGFILE-A1,REGFILE-A2,REGFILE-A3}
{
    \path (xInstr|-\y) node[circ] {};
}

% REGFILE.RD1 -> REGRD.in1
\draw[line data] (REGFILE-RD1) -- (REGRD.west|-yREGRD1);

% REGFILE.RD2 -> REGRD.in2
\draw[line data] (REGFILE-RD2) -- (REGRD.west|-yREGRD2);

% REGRD.out1 -> MUXSrcA.in3
\draw[line data] (REGRD.east|-yREGRD1) node[above right,signal] {A} -- (MUXSrcA.blpin 3);

% REGRD.out2 -> MUXSrcB.in3
\draw[line data] (REGRD.east|-yREGRD2)  node[above right,signal] {B} -- (MUXSrcB.blpin 1);

% REGRD.out2 -> MEM.WD
\draw[line data] (xWriteData|-yREGRD2) node[circ] {} -- node[left,pos=0.6,signal] {\rotatebox{-90}{WriteData}} (xWriteData|-yWriteDataWB) -- (xWriteDataWB|-yWriteDataWB) |- (MEM-WD);

% Extend ->[ImmExt] MUXSrcB.in2
\draw[line data] (IMMExtend.east) node[above right,signal] {ImmExt} -- (xIMMExt|-IMMExtend.east) |- (MUXSrcB.blpin 2);

% 4 -> MUXSrcB.in3
\draw[emph data] (xMUXSrcBin3|-MUXSrcB.blpin 3) node[left,line data] {4} node[ocirc,line data] {} -- node[below,pos=0.4,value] {4} (MUXSrcB.blpin 3);

% REGInstr.out2 -> MUXSrcA.in2
\draw[line data] (REGInstr.east|-yREGInstr1) node[above right,signal] {PCCurr} -- (xMUXSrcAin2|-yREGInstr1) |- (MUXSrcA.blpin 2);

% MUXSrcA ->[SrcA] ALU.in1
\draw[emph data] (MUXSrcA.brpin 1) -- node[below,value] {pc} (ALU.blpin 1) node[above left,signal] {SrcA}; 

% MUXSrcB ->[SrcB] ALU.in2
\draw[emph data] (MUXSrcB.brpin 1) -| (xSrc|-ALU.blpin 2) -- node[below,value] {4} (ALU.blpin 2) node[above left,signal] {SrcB};

% ALU.out -> REGALUOut
\draw[emph data no arrow] (ALU.brpin 2) node[above right,signal] {ALUResult} -- node[below,value] {pc+4} (xALUResult|-ALU.brpin 2); 
\draw[line data] (xALUResult|-ALU.brpin 2) -- (REGALUOut.west);

% REGALUout -> MUXResult.in1
\draw[line data] (REGALUOut.east) node[above right,signal] {ALUOut} -- (MUXResult.blpin 1);

% ALUResult -> MUXResult.in3
\draw[emph data] (xALUResult|-ALU.brpin 2) node[circ] {} -- (xALUResult|-MUXResult.blpin 3) -- node[below,value,pos=0.4] {pc+4} (MUXResult.blpin 3);

% REGData ->[Data] MUXResult.in2
\draw[line data] (REGData.east) node[above right,signal] {Data} -| (xData|-MUXResult.blpin 2) -- (MUXResult.blpin 2);

% MUXResult.out -> REGPC
\draw[emph data] (MUXResult.brpin 1) -| (xResult|-yResultWB) node[above left,signal] {Result} node[above left,yshift=2cm,value] {pc+4} -- (xREGPCin|-yResultWB) -- (xREGPCin|-yResultWB) -- node[right,signal] {\rotatebox{90}{PCNext}} (xREGPCin|-REGPC.west) node[above,value] {pc+4} -- (REGPC.west);

% Result -> MUXAdr.in2
\draw[line data] (xResultAdr|-yResultWB) node[circ] {} |- (MUXAdr.blpin 2);

% Result -> REGFILE.WD3
\draw[line data] (xResultWD3|-yResultWB) node[circ] {} |- (REGFILE-WD3);


% ---- Line Control ---- %

% RegWrite -> REGFILE.WE
\draw[line control] (CU-RegWrite) -| (REGFILE-WE) coordinate (xV-RegWrite);

% ImmSrc -> Extend
\draw[line control] (CU-ImmSrc) -| (xImmSrc|-yImmSrc) coordinate (xV-ImmSrc) -| (IMMExtend.north);

% ALUSrcB -> MUXSrcB
\draw[emph control] (CU-ALUSrcB) -| (MUXSrcB.btpin 1) coordinate (xV-ALUSrcB);

% ALUSrcA -> MUXSrcA
\draw[emph control] (CU-ALUSrcA) -| (MUXSrcA.btpin 1) coordinate (xV-ALUSrcA);

% ALUControl -> ALU
\draw[emph control] (CU-ALUControl) -| (ALU.btpin 1) coordinate (xV-ALUControl);

% ResultSrc -> MUXResult
\draw[emph control] (CU-ResultSrc) -| (MUXResult.btpin 1) coordinate (xV-ResultSrc);

% PCWrite -> REGPC.EN
\draw[emph control] (CU-PCWrite) -| (xREGPCEN|-yREGPCEN) coordinate (xV-PCWrite) -| (REGPC.south);

% AdrSrc -> MUXAdr
\draw[emph control] (CU-AdrSrc) -| (MUXAdr.btpin 1) coordinate (xV-AdrSrc);

% MemWrite -> MEM.WE
\draw[line control] (CU-MemWrite) -| (MEM-WE) coordinate (xV-MemWrite);

% IRWrite -> REGInstr.EN
\draw[emph control] (CU-IRWrite) -| (xREGInstrEN|-yREGInstrEN) coordinate (xV-IRWrite) -| (REGInstr.south);

% ALU.Zero -> ControlUnit.Zero
\draw[line control] (ALU.brpin 1) node[above right,signal] {Zero} -| (xZero|-yZeroCU) -| (xZeroCU|-CU-Zero) -- (CU-Zero);

% ---- Control Signal ---- %
\foreach \x/\val/\direc/\type in%
{%
    PCWrite/1/right/emph control,
    AdrSrc/0/right/emph control,
    MemWrite/0/right/line control,
    IRWrite/1/right/emph control,
    RegWrite/0/left/line control,
    ImmSrc/xx/left/line control,
    ALUSrcA/00/left/emph control,
    ALUSrcB/10/left/emph control,
    ALUControl/000/left/emph control,
    ResultSrc/10/left/emph control%
}
{
    \path[\type] (xV-\x |-CU-funct7) node[\direc,value] {\val};
}

% ---- Register Content ---- %
\path (REGPC) node[value,draw,fill=white] {\rotatebox{90}{pc\vphantom{hp}}};

% ---- Logo ---- %
\path (REGALUOut|-CU-ImmSrc) ++ (-0.7,-0.2) coordinate (LogoCenter);

\path (LogoCenter) ++(0.5,-0.3) node[minimum width=3.8cm,minimum height=5.5cm,fill=black!20!white] {};

\path (LogoCenter) node[draw,minimum width=3.8cm,minimum height=5.5cm,ultra thick,black,fill=black!50!white,align=center,text=white] {{\LARGE\bfseries\rmfamily S0\;:\;Fetch}\\ AdrSrc=0\\ IRWrite\\ ALUSrcA=00\\ ALUSrcB=10\\ ALUOp=00\\ ResultSrc=10\\ PCUpate};

\end{tikzpicture}
\end{document}