\section{缓存原理与存储架构}

\subsection{缓存原理}

想像一个图书馆,而你在为毕业论文搜集资料。内存就像图书馆里一排又一排的书架,它们真的有很多书,但是穿过一排排书架找到一本想要的书需要花费很多时间。因此,我们会将几本正在看的书放在手边的桌子上。缓存就像图书馆里的桌子,只能放下有限的几本书,但是不用离开座位就可以翻阅。缓存设计最关键的问题就在于,我们应该将什么样的书放在桌上可以最大程度的减少我们离开座位去找书的次数?第一个想法称为时间局域性,如果我需要翻阅拉扎维的模拟CMOS集成电路,在未来的几个小时,我或许还要再次翻阅其中的章节,毕竟这本书非常晦涩难懂\footnote{如果你恰巧在学习模拟集成电路,艾伦的《CMOS模拟集成电路设计》比拉扎维的《模拟CMOS集成电路设计》有条理许多!},那不妨将这本书在桌上多放一会儿,直到桌上没有更多空间了。第二个想法称为空间局域性,如果我们准备从书架上拿了拉扎维的模拟CMOS集成电路,或许我们也会对同一个书架上奥本海姆的信号与系统很感兴趣,那不妨将这两本书一起抱回桌上。
\begin{itemize}
    \item 时间局域性(Temporal Locatity):若一个数据被访问,很有可能会再次访问该数据。
    \item 空间局域性(Spital Locatity):若一个数据被访问,很有可能会访问其临近数据。
\end{itemize}

缓存(Cache)的设计就是在利用数据访问行为的两个规律:时间局域性、空间局域性。

\subsection{缓存与层次化存储}

缓存的想法其实和存储器在速度和容量上的折中优化有关:我们无法制造出又快又大容量的存储器!但是我们可以制造“快但容量小”或“慢但容量大”的存储器,速度由快至慢有
\begin{itemize}
    \item 静态随机存取存储器(Static Random Access Memory, SRAM)
    \item 动态随机存取存储器(Dyanamic Random Access Memory, DRAM)
    \item 固态硬盘(Solid State Disk, SSD)
    \item 机械硬盘(Hard Drive Disk, HDD)
\end{itemize}

缓存的想法可以概括为:在越靠近CPU的位置使用越快但容量越小的存储器,使最有可能被访问的数据能在最短的时间被访问到。第一层是缓存(Cache),速度最快,它通常由SRAM构成,这是由六个晶体管构成的双稳态电路,在通电的状态下可以持久保持数据。Cache自身往往还有若干层,一般有L1 Cache和L2 Cache两层,容量通常为数百$\si{\kibi\byte}$和数$\si{\mebi\byte}$,可以在大致$1$个和$10$个周期内完成数据访问。Cache会和CPU做在同一块芯片上,从而最小化互连线的延时。第二层是内存(Memory),速度其次,它通常由DRAM构成,这是由一个晶体管和一个电容构成的电路,在通电的状态下需要定期刷新数据,否则会因为电容漏电丢失数据,容量通常为数十$\si{\gibi\byte}$,可以在约$\si{100}$个周期内完成数据访问。除此之外,硬盘作为非易失性存储,除了长期保存数据外,在有些情况下也可以作为虚拟内存(Virtual Memory)以内存交换的方式加入这个层次化存储体系,当然,作为必要的代价,它的速度就非常慢了。

\begin{Figure}[存储架构]
    \includegraphics[width=\linewidth]{MemoryStructure.fig.pdf}
\end{Figure}

\subsection{缓存的访问时间}

设缓存和内存的访问时间分别是$t_{cache}$和$t_{memory}$,若访问的数据存在于缓存中,称为缓存命中(Cache Hit),若访问的数据不存在于缓存中,称为缓存失效(Cache Miss),此时,需要进一步去内存中读取数据,假设缓存失效发生的概率记为$M$,则访问时间可以表示为
\begin{BoxFormula}[缓存的访问时间]
    若考虑缓存失效,访问缓存中数据所需的时间应为
    \begin{Equation}
        t=t_{cache}+Mt_{memory}
    \end{Equation}
\end{BoxFormula}
