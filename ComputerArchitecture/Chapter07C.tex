\section{多周期处理器}

多周期处理器的完整电路如\cref{fig:多周期RISC-V处理器}所示。通过\cref{sec:单周期处理器}的学习,我们已经知道单周期处理器有以下三个缺点:第一个缺点是它需要分离的IMEM和DMEM,然而大部分处理器只有一块MEM用来存储指令和数据。第二个缺点是它的时钟周期需要支持最慢的\code{lw}指令,即便其他指令需要的时间更少。第三个缺点是它需要三个加法器:一个用于ALU,两个用于PC,而加法器是硬件资源开销相对大的电路单元。而多周期处理器提出了一种可以解决上述缺点的方案。
\begin{Figure}[多周期RISC-V处理器]
    \includegraphics[width=\linewidth]{Chapter07C_01.fig.pdf}
\end{Figure}

多周期处理器对于这些缺点的解决思路是
\begin{enumerate}
    \item 多周期处理器将一条指令拆分为若干更小的步骤,REGFILE、ALU、MEM这三个耗时最长的元件(事实上指令\code{lw}最慢的原因就是它三者全部使用了)在每个步骤只被允许使用一次,以保证每个步骤具有基本相等的延时。而根据复杂程度不同,可以为不同的指令分配不同的周期数,简单的指令会使用较少的周期,复杂的指令则使用较多的周期。
    \item 多周期处理器将IMEM和DMEM合并,指令的读取和数据的读写在不同周期进行。
    \item 多周期处理器将两个PC加法器合并至ALU,其在不同周期以不同目的复用。
    \item 多周期处理器需要添加一些非架构寄存器,在周期间保存中间结果。
\end{enumerate}
比较\cref{fig:多周期RISC-V处理器}和\cref{fig:单周期RISC-V处理器},我们可以看出,多周期处理器相较于单周期处理器的主要改动包括:首先由于ALU合并了两个PC加法器,其\code{SrcA}和\code{SrcB}需要增加额外的MUX通道。现在ALU需要处理的计算除了\code{rs1+rs2}和\code{rs1+imm}还包含\code{pc+4}和\code{pc+imm},因此,\code{SrcB}增加了\code{4}的输入,\code{SrcA} 增加了两个不同来源的\code{pc}的输入,一个直接来自\code{PC}寄存器,一个来自其寄存之后的\code{PCCurr},因为有些情况下我们需要立即得到这个信号,无法等到下个周期,这样的设计在多周期处理器中是很多见的。其次由于MEM合并了IMEM和DMEM,在MEM的地址输入端需要增加一个MUX确定地址是指令地址(来自\code{PC})还是数据地址(来自\code{ALU})。最后输出端的\code{Result}现在被接到REGFILE写回、MEM地址、PC寄存器三个位置,以适应电路单元复用所需的额外数据通路。同时要注意的是,尽管产生\code{Result}的MUX仍然是三通道,但是内容和单周期处理器不一样,现在三个通道分别是ALU的计算结果\code{ALUResult}和其寄存一拍后的\code{ALUOut}(新增,原先单周期的\code{PCPlus4}不再需要),以及MEM读出的\code{Data}。

\subsection{多周期RISC-V处理器的控制单元}
多周期RISC-V处理器的控制单元如\cref{fig:多周期RISC-V处理器的有限状态机}所示,注意到最大的变化是Main Decoder变成了一个状态机Main FSM。这是因为多周期处理器中,指令需要多个周期才能执行完成,大部分控制信号在同一个指令的不同周期间是需要变化的,故需要一个状态机来支持。当然,有一个例外是\code{ImmSrc}信号,由于立即数的扩展模式在同一指令的执行中不变,故置于状态机之外。
\begin{Figure}[多周期RISC-V处理器的控制单元]
    \includegraphics[scale=0.8]{Chapter07C_13.fig.pdf}
\end{Figure}
% 通过两个控制信号的“重命名”,我们可以洞见从单周期处理器到多周期处理器的\code{PC}寄存器更新方式的差别。这两个“重命名”分别是\code{Jump->PCUpdte}和\code{PCSrc->PCWrite}。单周期处理器中,\code{PCSrc}控制MUX在\code{pc}

\cref{fig:多周期RISC-V处理器}展示了多周期处理器控制单元的有限状态机的状态转移逻辑。简洁起见,并不是每一个控制信号都被标出,我们遵循这样的约定:如\code{ALUOp=00}这样显式指明赋值的控制信号,在其未出现的状态中被认为是不重要的,可以取任意值(即\code{x})。不同的是,如\code{PCUpdte}这样仅名称出现的控制信号(均为$\qnum{1}{bit}$),在其出现的状态中代表赋\code{1},在其未出现的状态中代表赋\code{0}。

\begin{Figure}[多周期RISC-V处理器的有限状态机]
    \includegraphics[width=\linewidth]{Chapter07C_14.fig.pdf}
\end{Figure}

\cref{tab:Main FSM各指令的状态转移过程}展示了各个指令类型的状态转移过程。S0和S1是所有指令都需经历的,LOAD类型需要经历最多的$5$个状态,BRANCH类型则只需最少的$3$个状态,大部分都是$4$个状态。

\begin{Table}[Main FSM各指令的状态转移过程]!!
    \begin{tblr}
    {
        colspec={Xc*{5}{l}},
        cell{1}{3}={c=5}{},
        column{1}={rightsep=2pt},
        column{3}={rightsep=2pt},
        column{4-7}={leftsep=2pt,rightsep=2pt},
    }
        类型&状态数&状态转移路径&&&&\\
        I/LOAD&5&S0(Fetch)&S1(Decode)&S2(MemAdr)&S7(MemRead)&S10(MemWB)\\
        S/STORE&4&S0(Fetch)&S1(Decode)&S2(MemAdr)&S8(MemWrite)\\
        R/OP&4&S0(Fetch)&S1(Decode)&S3(ExcuteR)&S8(ALUWB)\\
        I/OP-IMM&4&S0(Fetch)&S1(Decode)&S4(ExcuteI)&S8(ALUWB)\\
        J/JAL&4&S0(Fetch)&S1(Decode)&S5(Jump)&S8(ALUWB)\\
        B/BRANCH&3&S0(Fetch)&S1(Decode)&S6(Branch)\\
    \end{tblr}
\end{Table}

\cref{tab:Main FSM的真值表}以表格形式展示了\cref{fig:多周期RISC-V处理器}的状态机的各个状态下,所有控制信号的取值。

\begin{Table}[Main FSM的真值表]!!
    \begin{tblr}
    {
        colspec={X[c]*{10}{c}},
        cell{1-Z}{2-Z}={preto=\ttfamily},
        column{2-Z}={leftsep=3pt,rightsep=3pt},
    }
        状态&AdrSrc&ALUSrcA&ALUSrcB&ALUOp&ResultSrc&MemWrite&RegWrite&IRWrite&PCUpdte&Branch\\
        S0 & 0 & 00 & 10 & 00 & 10 & 0 & 0 & 1 & 1 & 0\\
        S1 & x & 01 & 01 & 00 & xx & 0 & 0 & 0 & 0 & 0\\ %hline%
        S2 & x & 10 & 01 & 00 & xx & 0 & 0 & 0 & 0 & 0\\ 
        S3 & x & 10 & 00 & 10 & xx & 0 & 0 & 0 & 0 & 0\\ 
        S4 & x & 10 & 01 & 10 & xx & 0 & 0 & 0 & 0 & 0\\ 
        S5 & x & 01 & 10 & 00 & 00 & 0 & 0 & 0 & 1 & 0\\
        S6 & x & 10 & 00 & 01 & 00 & 0 & 0 & 0 & 0 & 1\\ %hline%
        S7 & 1 & xx & xx & xx & 00 & 0 & 0 & 0 & 0 & 0\\
        S8 & 1 & xx & xx & xx & 00 & 1 & 0 & 0 & 0 & 0\\ %hline%
        S9 & x & xx & xx & xx & 00 & 0 & 1 & 0 & 0 & 0\\
        S10& x & xx & xx & xx & 01 & 0 & 1 & 0 & 0 & 0\\
    \end{tblr}
\end{Table}

\subsection{多周期处理器的数据通路}
多周期处理器的分析难点在于,一条指令的执行需要经历若干状态,而每个状态都意味着一张不同的数据通路,有些状态又会被不同指令复用。\cref{tab:多周期RISC-V处理器的数据通路}展示了S0至S10的数据通路。

接下来,我们会逐一分析多周期处理器每个状态的数据通路。
\begin{Table}[多周期RISC-V处理器的数据通路]!!
    \begin{tblr}
    {
        colspec={X[c]X[c]},
        hline{2}={\linenone},
        hline{odd[2-Y]}={\linethin},
        vline{1,Z}={\linethick},
        vline{2}={\linethin},
        row{odd[1-Z]}={abovesep=6pt,belowsep=-4pt},
        row{even[1-Z]}={abovesep=0pt},
        cell{1-2}{1}={c=2}{halign=c},
        rowhead=0,
    }
        \includegraphics[width=0.45\textwidth]{Chapter07C_02.fig.pdf}&\\*
        S0 : Fetch&\\
        % ----------------%
        \includegraphics[width=0.45\textwidth]{Chapter07C_03.fig.pdf}&
        \includegraphics[width=0.45\textwidth]{Chapter07C_04.fig.pdf}\\*
        S1 : Decode&S2 : MemAdr\\
        % ----------------%
        \includegraphics[width=0.45\textwidth]{Chapter07C_05.fig.pdf}&
        \includegraphics[width=0.45\textwidth]{Chapter07C_06.fig.pdf}\\*
        S3 : ExcuteR&S4 : ExcuteI\\
        % ----------------%
        \includegraphics[width=0.45\textwidth]{Chapter07C_07.fig.pdf}&
        \includegraphics[width=0.45\textwidth]{Chapter07C_08.fig.pdf}\\*
        S5 : Jump&S6 : Branch\\
        % ----------------%
        \includegraphics[width=0.45\textwidth]{Chapter07C_09.fig.pdf}&
        \includegraphics[width=0.45\textwidth]{Chapter07C_10.fig.pdf}\\*
        S7 : MemRead&S8 : MemWrite\\
        % ----------------%
        \includegraphics[width=0.45\textwidth]{Chapter07C_11.fig.pdf}&
        \includegraphics[width=0.45\textwidth]{Chapter07C_12.fig.pdf}\\*
        S9 : WBALU&S10 : WBMem\\
        % ----------------%
    \end{tblr}
\end{Table}

\paragraph{状态S0(Fetch)}
状态S0(Fetch)的主要功能是根据当前指令地址\code{pc}取出指令\code{instr}并计算顺序执行的下一指令地址\code{pc+4}。该状态从\code{pc}从PC寄存器中取出开始,它会经历三条并行路径。第一条路径是\code{pc}通过MUX到达\code{Adr}即MEM的地址端,从MEM中取出指令内容\code{instr}并写入寄存器\code{Instr}。第二条路径是\code{pc}直接写入寄存器\code{PCCurr}。应指出,后续寄存器\code{PCCurr}和\code{Instr}会由于\code{IRWrite}信号的撤去使写入的\code{pc}和\code{instr}被锁存住,在整个指令执行期间始终保持可用。第三条路径是\code{pc}通过MUX到达\code{SrcA},同时\code{SrcB}则为\code{4},两者经过ALU得到\code{pc+4},作为\code{ALUResult}通过MUX到达\code{Result}并写回\code{PC}寄存器,作为下一指令的地址。若之后发现执行的是分支指令或跳转指令,且确实需要跳转,会再次用跳转地址覆写\code{PC}寄存器。

这里有必要说明的是\code{SrcA}和\code{Result}前的两个MUX,它们都对相似的信号有两个通道
\begin{itemize}
    \item 产生\code{SrcA}的MUX的\code{00,01}通道分别是\code{PC}和\code{PCCurr},为何需要为\code{pc}保留两条通路呢?首先,虽说\code{PCCurr}寄存器在这一步就是写入了\code{PC}寄存器中读出的\code{pc},然而,寄存器的写入需要等待下一周期的上升沿才生效,所以必须得保留一条连接到\code{PC}寄存器的快速通道,以保证当前周期\code{pc}就能送到ALU。另一方面,由于\code{PC}寄存器在该状态后就会被写入下一指令的地址,因此后续如果再需要\code{pc},就得从\code{PCCurr}寄存器获取了。
    \item 产生\code{Result}的MUX的\code{00,01}通道分别是\code{ALUOut}和\code{ALUResult},后者是ALU的计算结果,前者是ALU的计算结果寄存一拍的结果。因为在有些情况下对ALU计算结果的使用过于耗时,以至于我们需要先将ALU计算结果存起来,在下一周期继续使用。
\end{itemize}

\paragraph{状态S1(Decode)}
状态S1(Decode)的主要功能是完成指令解码,但同时,它会计算\code{pc+imm}作为假如进行跳转的下一指令地址。首先,指令内容\code{instr}从寄存器\code{Instr}发出,指令各字段被分拆进入不同的单元。扩展字段\code{op,funct3,funct7}进入控制单元,确定后续的状态转移路径,并产生\code{ImmSrc}信号,寄存器字段\code{xrs1,xrs2}进入REGFILE,读出寄存器数据\code{rs1,rs2}并写入寄存器\code{A,B}。立即数字段\code{ximm}进入Extend并在\code{ImmSrc}的控制下产生立即数\code{imm},不过并没有写入寄存器,但由于整个执行过程中\code{instr}始终有效,故不妨碍。立即数\code{imm}通过MUX到达\code{SrcB},而同时,\code{pc}从\code{PCCurr}寄存器读出通过MUX到达\code{SrcA},两者经过ALU得到\code{pc+imm},该结果会先写入\code{ALUOut}寄存器。当然,对于大部分与跳转无关的指令,该处进行的计算看上去毫无意义,但不妨碍我们先做好准备。若不涉及跳转,后续会抛弃这个结果。

\paragraph{状态S2(MemAdr)、S3(ExcuteR)、S4(ExcuteI)、S5(Jump)、S6(Branch)}
状态S2(MemAdr)、S3(ExcuteR)、S4(ExcuteI)、S5(Jump)、S6(Branch)是五个平级的状态,它们的共同点是,各自完成了其指令类型中最重要的一次ALU计算

\begin{itemize}
    \item 状态S2(MemAdr)进行加法计算,得到MEM地址\code{imm(rs1)}。
    \item 状态S3(ExcuteR)和S4(ExcuteI)对\code{rs1,rs2}或\code{rs1,imm}进行指定的ALU运算。
    \item 状态S5(Jump)进行加法计算,得到用于写回REGFILE的\code{pc+4}。
    \item 状态S6(Branch)进行减法计算,通过\code{rs1-rs2}是否为零判断两者是否相等。
\end{itemize}

除了S6(Branch)之外,其余的ALU结果都会写入寄存器\code{ALUOut}等待下一周期的使用,其中,我们将S3(ExcuteR)、S4(ExcuteI)、S5(Jump)的ALU结果统一记为\code{rd},因为这三者的结果最终都是用于写回REGFILE,作为对比,S2(MemAdr)的ALU结果\code{imm(rs1)}则是作为MEM地址。对于S6(Branch),其关注ALU产生的\code{Zero}信号,这会送至控制单元。

在这些状态中,状态S5(Jump)和S6(Branch)还需要额外关心的一点是,是否要用当前周期寄存器\code{ALUOut}存储的跳转地址\code{pc+imm}覆盖寄存器\code{PC}中存储的顺序地址\code{pc+4}。参照控制单元\cref{fig:多周期RISC-V处理器的控制单元}的内部电路,对于状态S6(Branch),状态机输出\code{Branch=1},当同时收到\code{Zero=1}时更新指令地址。对于状态S5(Jump),状态机输出\code{PCUpdate=1}更新指令地址。值得注意的是,之所以这里\code{PCUpdate}不对应称为\code{Jump},是因为S0(Fetch)写入\code{pc+4}时也是通过该信号进行的,其本质是进行无条件的指令地址更新,而不仅仅用于跳转时。除此之外,很有趣的是,尽管\code{pc+4}在S0(Fetch)已经被计算,但是S5(Jump)中又计算了一遍\code{pc+4},作为写回REGFILE的数据。重复计算听上去很愚蠢,但这也同时避免了更多状态的添加。

在这里还有一个细节,上述五个状态中,只有S2(MemAdr)标注了寄存器\code{B}中刷新\code{rs2}的通路,这意味着其后续状态仍会使用\code{rs2}。它在S8(MemWrite)作为写入MEM的数据。

\paragraph{状态S7(MemRead)、S8(MemWrite)}
状态S7(MemRead)、S8(MemWrite)分别完成了存储器的读和写。两者大部分数据通路是相同的,寄存器\code{ALUOut}存储的\code{imm(rs1)}通过MUX到达\code{Result},控制信号\code{AdrSrc=1}使MEM接受\code{Result}而不是\code{PC}作为地址,从而使\code{imm(rs1)}送至MEM地址端。不同之处在于,对于S7(MemRead),\code{MemWrite=0} 使数据\code{[imm(rs1)]}自MEM被读出,并写入寄存器\code{Data},遵照前例,也记为\code{rd}。对于S8(MemWrite),在寄存器\code{B}中的\code{rs2}被送至MEM的写数据端,同时,\code{MemWrite=1} 使MEM进行写状态,使\code{rs2}被写入地址\code{imm(rs1)}处。

\paragraph{状态S9(WBALU)、S10(WBMem)}
状态S9(WBALU)、S10(WBMem)完成了对寄存器堆的写回。两者唯一的区别是\code{Result}的数据源,S9(WBALU)的源是\code{ALUOut}寄存器,S10(WBMem)的源是\code{Data}寄存器。待写回的数据\code{rd}通过MUX到达\code{Result}后连接到REGFILE的写数据端,同时\code{instr}中代表目标寄存器地址的字段\code{xrd}连接到REGFILE的写地址端,\code{RegWrite=1} 将数据写入。

\subsection{多周期处理器的性能分析}
多周期处理器可用的最短周期同样取决于最耗时的状态,这有两个潜在的可能。

第一个是状态S0(Fetch),它从一个$t_{pcq}$开始,该时间内有两件事在同时发生,包括\code{pc}从\code{PC}寄存器中读出和控制单元中状态机新状态的读出,随后,在\code{pc}通过MUX前,由于MUX需要控制信号,要先等待$t_{dec}$的控制单元译码时间,随后是\code{pc}通过MUX的$t_{mux}$和ALU计算\code{pc+4}的$t_{alu}$,最后是\code{pc+4}通过MUX的$t_{mux}$以及写入\code{PC}寄存器需预留的$t_{setup}$
\begin{Equation}
    T_{c,multi,s1}=t_{pcq}+t_{dec}+t_{mux}+t_{alu}+t_{mux}+t_{setup}
\end{Equation}
第二个是状态S7(MemRead),它同样从$t_{pcq}$和$t_{dec}$开始,地址\code{imm(rs1)}从\code{ALUOut}寄存器读出,随后是顺次通过\code{Result}和\code{Adr}前的两个MUX的$2t_{mux}$和从MEM中读取数据\code{[imm(rs1)]}所需的$t_{mem}$,最后是数据\code{[imm(rs1)]}写入\code{Data}寄存器所需预留的$t_{setup}$
\begin{Equation}
    T_{c,multi,s7}=t_{pcq}+t_{dec}+2t_{mux}+t_{mem}+t_{setup}
\end{Equation}

综合两式,即有
\begin{Equation}
    T_{c,multi}=t_{pcq}+t_{dec}+2t_{mux}+\max(t_{mem},t_{alu})+t_{setup}
\end{Equation}
而根据\cref{tab:参考电路延时},注意到MEM延时$t_{mem}=\qnum{200}{ps}$比ALU延时$t_{alu}=120{ps}$更长
\begin{Equation}
    T_{c,multi}=t_{pcq}+t_{dec}+2t_{mux}+t_{mem}+t_{setup}
\end{Equation}
整理如下,并计算出数值
\begin{BoxFormula}[多周期处理器的最小周期]
    多周期处理器的最小周期是
    \begin{Equation}
        T_{c,multi}=t_{pcq}+t_{dec}+2t_{mux}+t_{mem}+t_{setup}=\qnum{375}{ps}
    \end{Equation}
\end{BoxFormula}
多周期处理器的$\te{CPI}$需要结合\cref{tab:参考指令类型比例}和\cref{tab:Main FSM各指令的状态转移过程}做加权平均得出
\begin{BoxFormula}[多周期处理器的指令平均周期数]
    多周期处理器的CPI是
    \begin{Equation}
        \te{CPI}_{multi}=4.14
    \end{Equation}
\end{BoxFormula}
多周期处理器在指令数$n=\qnum{1e11}{}$下的程序执行时间$T=n\cdot\te{CPI}\cdot T_{c}$因而为
\begin{BoxFormula}[多周期处理器的程序执行时间]
    多周期处理器的程序执行时间是
    \begin{Equation}
        T_{multi}=\qnum{155}{s}
    \end{Equation}
\end{BoxFormula}
若比较\cref{fml:多周期处理器的程序执行时间}和\cref{fml:单周期处理器的程序执行时间},令人啼笑皆非的是,多周期处理器大费周章试图令简单的指令能更快执行,最终总的执行时间却比单周期处理器高了一倍。这有多方面原因,首先,多周期处理器为了更大程度的复用电路单元,数据通路上需要的MUX更多,这占用了一定时间。其次,理论上将一条指令拆分为$5$个步骤可以令周期变为原来的$1/5$,而有些指令只要$3,4$个步骤,那么还能赚些时间。但实际情况是,周期减小最终会受制于寄存器传播时间和准备时间之和$t_{pcq}+t_{setup}$,这是同步时序能工作的极限。多周期处理器将指令拆分后,每个小周期都会需要$t_{pcq}+t_{setup}$,而$t_{pcq}+t_{setup}$又占到了$T_{c,multi}$中相当一部分,这就抵消了其优化。
